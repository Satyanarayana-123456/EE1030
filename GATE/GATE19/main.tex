\let\negmedspace\undefined
\let\negthickspace\undefined
\documentclass[journal]{IEEEtran}
\usepackage[a5paper, margin=10mm, onecolumn]{geometry}
%\usepackage{lmodern} % Ensure lmodern is loaded for pdflatex
\usepackage{tfrupee} % Include tfrupee package

\setlength{\headheight}{1cm} % Set the height of the header box
\setlength{\headsep}{0mm}     % Set the distance between the header box and the top of the text

\usepackage{gvv-book}
\usepackage{gvv}
\usepackage{cite}
\usepackage{amsmath,amssymb,amsfonts,amsthm}
\usepackage{algorithmic}
\usepackage{graphicx}
\usepackage{textcomp}
\usepackage{xcolor}
\usepackage{txfonts}
\usepackage{listings}
\usepackage{enumitem}
\usepackage{mathtools}
\usepackage{gensymb}
\usepackage{comment}
\usepackage[breaklinks=true]{hyperref}
\usepackage{tkz-euclide} 
\usepackage{listings}
% \usepackage{gvv}                                        
\def\inputGnumericTable{}                                 
\usepackage[latin1]{inputenc}                                
\usepackage{color}                                            
\usepackage{array}                                            
\usepackage{longtable}                                       
\usepackage{calc}                                             
\usepackage{multirow}                                         
\usepackage{hhline}                                           
\usepackage{ifthen}                                           
\usepackage{lscape}
\begin{document}

\bibliographystyle{IEEEtran}
\vspace{3cm}




\title{
%	\logo{
GATE - 2019 - CE

\large{EE1030 : Matrix Theory}

Indian Institute of Technology Hyderabad
%	}
}
\author{Satyanarayana Gajjarapu

AI24BTECH11009
}	





\maketitle




\bigskip

\renewcommand{\thefigure}{\theenumi}
\renewcommand{\thetable}{\theenumi}


\section{14 - 26}


\begin{enumerate}
\item For a small value of $h$, the Taylor series expansion for $f\brak{x+h}$ is
\begin{enumerate}
    \item $f\brak{x} + hf'\brak{x} + \frac{h^2}{2!}f''\brak{x} + \frac{h^3}{3!}f'''\brak{x} + \cdots + \infty$
    \item $f\brak{x} - hf'\brak{x} + \frac{h^2}{2!}f''\brak{x} - \frac{h^3}{3!}f'''\brak{x} + \cdots + \infty$
    \item $f\brak{x} + hf'\brak{x} + \frac{h^2}{2}f''\brak{x} + \frac{h^3}{3}f'''\brak{x} + \cdots + \infty$
    \item $f\brak{x} - hf'\brak{x} + \frac{h^2}{2}f''\brak{x} - \frac{h^3}{3}f'''\brak{x} + \cdots + \infty$ \\
\end{enumerate}
\item A plane truss is shown in the figure \brak{not\ drawn\ to\ scale}.
\begin{figure}[!ht]
\centering
\resizebox{1\textwidth}{!}{%
\begin{circuitikz}
\tikzstyle{every node}=[font=\normalsize]
\draw [->, >=Stealth] (6.25,9.5) -- (6.25,11);
\draw [->, >=Stealth] (6.25,9.5) -- (14.5,9.5);
\draw [short] (6.25,9.25) -- (6.25,8.75);
\draw [short] (12.75,9.25) -- (12.75,8.75);
\draw [<->, >=Stealth] (6.25,9) -- (12.75,9);
\draw [line width=1.6pt, short] (6.25,9.5) -- (12.75,9.5);
\draw [->, >=Stealth] (2.5,10.75) -- (5,10.75);
\draw [->, >=Stealth] (2.5,10) -- (5,10);
\draw [->, >=Stealth] (2.5,9.25) -- (5,9.25);
\draw [->, >=Stealth] (2.5,8.5) -- (5,8.5);
\node [font=\normalsize] at (9.25,8.75) {0.25 m};
\node [font=\normalsize] at (14.75,9.5) {$x$};
\node [font=\normalsize] at (6.25,11.25) {$y$};
\node [font=\normalsize] at (3.5,11) {$U_{\infty}$};
\end{circuitikz}

}%
\end{figure}\\
Which one of the options contains ONLY zero force members in the truss ?
\begin{enumerate}
    \item FG, FI, HI, RS
    \item FG, FH, HI, RS
    \item FI, HI, PR, RS
    \item FI, FG, RS, PR \\
\end{enumerate}
\item An element is subjected to biaxial normal tensile strains of 0.0030 and 0.0020. The normal strain in the plane of maximum shear strain is
\begin{enumerate}
    \item Zero
    \item 0.0010
    \item 0.0025
    \item 0.0050 \\
\end{enumerate}
\item Consider the pin-jointed plane truss shown in the figure \brak{not\ drawn\ to\ scale}. Let $R_P$, $R_Q$, and $R_R$ denote the vertical reactions (upward positive) applied by the supports at P, Q, and R, respectively, on the truss. The correct combination of $\brak{R_P, R_Q, R_R}$ is represented by
\begin{figure}[!ht]
\centering
\resizebox{0.7\textwidth}{!}{%
\begin{circuitikz}
\tikzstyle{every node}=[font=\normalsize]
\draw [short] (5,10.5) -- (5,9.25);
\draw [short] (5,9.25) -- (11.5,9.25);
\draw [short] (11.5,10.5) -- (11.5,9.25);
\draw [short] (5,7.5) -- (11.5,7.5);
\draw [short] (11.5,7.5) -- (11.5,6.25);
\draw [short] (5,7.5) -- (5,6.25);
\draw [->, >=Stealth] (5,9) -- (6.25,9);
\draw [->, >=Stealth] (5,8.75) -- (6.25,8.75);
\draw [->, >=Stealth] (5,8.25) -- (6.25,8.25);
\draw [->, >=Stealth] (5,7.75) -- (6.25,7.75);
\draw [->, >=Stealth] (5,8) -- (6.25,8);
\draw [->, >=Stealth] (5,8.5) -- (6.25,8.5);
\draw [short] (11.5,9.25) .. controls (14,8.5) and (12.75,8) .. (11.5,7.5);
\draw [->, >=Stealth] (1.5,8) -- (1.5,9.75);
\draw [->, >=Stealth] (1.5,8) -- (3,8);
\draw [->, >=Stealth] (11.5,9) -- (12,9);
\draw [->, >=Stealth] (11.5,8.75) -- (12.5,8.75);
\draw [->, >=Stealth] (11.5,8.5) -- (12.75,8.5);
\draw [->, >=Stealth] (11.5,8.25) -- (12.75,8.25);
\draw [->, >=Stealth] (11.5,8) -- (12.5,8);
\draw [->, >=Stealth] (11.5,7.75) -- (12.25,7.75);
\draw [line width=0.2pt, short] (11.5,9.25) -- (11.5,7.5);
\draw [short] (5,9.25) -- (6.5,10.5);
\draw [short] (6,9.25) -- (7.5,10.25);
\draw [short] (7.25,9.25) -- (8.5,10.25);
\draw [short] (8.5,9.25) -- (9.75,10.25);
\draw [short] (9.75,9.25) -- (10.75,10.25);
\draw [short] (10.75,9.25) -- (11.5,10);
\draw [short] (5.5,9.25) -- (7,10.5);
\draw [short] (6.75,9.25) -- (8,10.25);
\draw [short] (8,9.25) -- (9.25,10.5);
\draw [short] (9.25,9.25) -- (10.25,10.25);
\draw [short] (10.25,9.25) -- (11.25,10.25);
\draw [short] (5,7) -- (5.75,7.5);
\draw [short] (5,6.5) -- (6.75,7.5);
\draw [short] (6,6.75) -- (7.5,7.5);
\draw [short] (7.25,6.5) -- (8.75,7.5);
\draw [short] (8.25,6.5) -- (9.5,7.5);
\draw [short] (9,6.25) -- (10,7.5);
\draw [short] (10.75,6.5) -- (11.5,7.5);
\draw [short] (10,6.5) -- (10.75,7.5);
\draw [short] (6.5,6.5) -- (8,7.5);
\node [font=\normalsize] at (12.25,9.5) {$y = y_0$};
\node [font=\normalsize] at (12.25,7.25) {$y = 0$};
\node [font=\normalsize] at (4.5,8.5) {$U_0$};
\node [font=\normalsize] at (3,7.75) {$x$};
\node [font=\normalsize] at (1.75,10) {$y$};
\end{circuitikz}

}%
\end{figure}
\begin{enumerate}
    \item \brak{30, -30, 30}kN
    \item \brak{20, 0, 10}kN
    \item \brak{10, 30, -10}kN
    \item \brak{0, 60, -30}kN \\
\end{enumerate}
\item Assuming that there is no possibility of shear buckling in the web, the maximum reduction permitted by IS 800-2007 in the (low-shear) design bending strength of a semi-compact steel section due to high shear is
 \begin{enumerate}
    \item zero
    \item 25\%
    \item 50\%
    \item governed by the area of the flange \\
\end{enumerate}
\item In the reinforced beam section shown in the figure \brak{not\ drawn\ to\ scale}, the nominal cover provided at the bottom of the beam as per IS 456-2000, is
\begin{figure}[!ht]
\centering
\resizebox{0.5\textwidth}{!}{%
\begin{circuitikz}
\tikzstyle{every node}=[font=\normalsize]
\draw [short] (3.5,11.25) -- (3.5,9);
\draw [short] (3.25,11.25) -- (3.25,9);
\draw [short] (3.25,11.25) -- (3.5,11.25);
\draw [short] (3.25,9) -- (3.5,9);
\draw [short] (3.25,11.25) -- (3,11);
\draw [short] (3.25,11) -- (3,10.75);
\draw [short] (3.25,10.75) -- (3,10.5);
\draw [short] (3.25,10.5) -- (3,10.25);
\draw [short] (3.25,10.25) -- (3,10);
\draw [short] (3.25,10) -- (3,9.75);
\draw [short] (3.25,9.75) -- (3,9.5);
\draw [short] (3.25,9.5) -- (3,9.25);
\draw [short] (3.25,9.25) -- (3,9);
\draw  (3.5,10.25) rectangle (11,10);
\draw  (11,11.25) rectangle (11.25,9);
\draw [short] (11.25,11.25) -- (11.5,11);
\draw [short] (11.25,11) -- (11.5,10.75);
\draw [short] (11.25,10.5) -- (11.5,10.25);
\draw [short] (11.25,10.75) -- (11.5,10.5);
\draw [short] (11.25,10) -- (11.5,9.75);
\draw [short] (11.25,10.25) -- (11.5,10);
\draw [short] (11.25,9.75) -- (11.5,9.5);
\draw [short] (11.25,9.5) -- (11.5,9.25);
\draw [short] (11.25,9.25) -- (11.5,9);
\draw [->, >=Stealth] (7.25,11.75) -- (7.25,10.25);
\draw [short] (5.75,11) -- (5.75,10.5);
\draw [short] (8.75,11) -- (8.75,10.5);
\draw [<->, >=Stealth] (3.5,10.75) -- (5.75,10.75);
\draw [<->, >=Stealth] (5.75,10.75) -- (7.25,10.75);
\draw [<->, >=Stealth] (7.25,10.75) -- (8.75,10.75);
\draw [<->, >=Stealth] (8.75,10.75) -- (11,10.75);
\draw [short] (5.75,10) -- (5.5,9.75);
\draw [short] (5.75,10) -- (6,9.75);
\draw [short] (5.25,9.75) -- (6.25,9.75);
\draw [short] (8.75,10) -- (8.5,9.75);
\draw [short] (8.75,10) -- (9,9.75);
\draw [short] (8.25,9.75) -- (9.25,9.75);
\draw [short] (5.25,9.75) -- (5.5,9.5);
\draw [short] (5.5,9.75) -- (5.75,9.5);
\draw [short] (5.75,9.75) -- (6,9.5);
\draw [short] (6,9.75) -- (6.25,9.5);
\draw [short] (8.5,9.75) -- (8.75,9.5);
\draw [short] (8.75,9.75) -- (9,9.5);
\draw [short] (9,9.75) -- (9.25,9.5);
\draw [short] (8.25,9.75) -- (8.5,9.5);
\draw [->, >=Stealth] (8,8.25) -- (7.25,10);
\draw [->, >=Stealth] (2.5,9.5) .. controls (1.75,10.25) and (2,10.25) .. (2.5,11) ;
\node [font=\large] at (2.5,11.25) {$M$};
\node [font=\normalsize] at (4.75,11) {3 m};
\node [font=\normalsize] at (6.5,11) {1 m};
\node [font=\normalsize] at (8,11) {1 m};
\node [font=\normalsize] at (9.75,11) {3 m};
\node [font=\normalsize] at (7.25,12) {20 kN};
\node [font=\normalsize] at (4.25,9.75) {$EI$};
\node [font=\normalsize] at (6.75,9.75) {$EI$};
\node [font=\normalsize] at (10,9.75) {$EI$};
\node [font=\normalsize] at (8,8) {Internal hinge};
\draw [short] (7,10) -- (7.5,10.25);
\draw [short] (7,10.25) -- (7.5,10);
\end{circuitikz}

}%
\end{figure}
\pagebreak
\begin{enumerate}
    \item 30 mm
    \item 36 mm
    \item 42 mm
    \item 50 mm \\
\end{enumerate}
\item The interior angles of four triangles are given below:
\begin{table}[h!]
  \centering
  \begin{circuitikz}
\tikzstyle{every node}=[font=\LARGE]
\draw  (5.5,11.75) rectangle (7.5,9.75);
\draw [short] (5.5,11.75) -- (7.5,9.75);
\draw [->, >=Stealth] (6.5,12.75) -- (6.5,11.75);
\draw [->, >=Stealth] (7.5,10.75) -- (9,10.75);
\draw [->, >=Stealth] (6.5,8.75) -- (6.5,9.75);
\draw [->, >=Stealth] (5.5,10.75) -- (4.25,10.75);
\draw [->, >=Stealth] (6.5,10.75) -- (7,11.25);
\draw [short] (6.75,11) -- (7,10.75);
\draw [short] (7,10.75) -- (6.75,10.5);
\draw [<->, >=Stealth] (6.75,9.75) .. controls (6.5,10.25) and (6.5,10.25) .. (7,10.25);
\node [font=\normalsize] at (6.5,13) {100 MPa};
\node [font=\normalsize] at (9.75,10.75) {100 MPa};
\node [font=\normalsize] at (6.5,8.5) {100 MPa};
\node [font=\normalsize] at (3.5,10.75) {100 MPa};
\node [font=\normalsize] at (6.5,10.4) {45\degree};
\node [font=\normalsize] at (7,11.5) {$\sigma_n$};
\end{circuitikz}

\end{table}\\
Which of the triangles are ill-conditioned and should be avoided in Triangulation surveys ?
\begin{enumerate}
   \item Both P and R 
   \item Both Q and R 
   \item Both P and S
   \item Both Q and S \\
\end{enumerate}
\item The coefficient of average rolling friction of a road is $f_r$ and its grade is +$G$\%. If the grade of this road is doubled, what will be the percentage change in the braking distance (for the design vehicle to come to a stop) measured along the horizontal (assume all other parameters are kept unchanged) ?
\begin{enumerate}
    \item $\frac{0.01 G}{f_r + 0.02 G} \times 100$
    \item $\frac{f_r}{f_r + 0.02 G} \times 100$
    \item $\frac{0.02 G}{f_r + 0.01 G} \times 100$
    \item $\frac{2 f_r}{f_r + 0.01 G} \times 100$ \\
\end{enumerate}
\item An isolated concrete pavement slab of length $L$ is resting on a frictionless base. The temperature of the top and bottom fibre of the slab are $T_t$ and $T_b$, respectively. Given: the coefficient of thermal expansion = $\alpha$ and the elastic modulus = $E$. Assuming $T_t > T_b$ and the unit weight of concrete as zero, the maximum thermal stress is calculated as
\begin{enumerate}
    \item $L \alpha \brak{T_t - T_b}$
    \item $E \alpha \brak{T_t - T_b}$
    \item $\frac{E \alpha \brak{T_t - T_b}}{2}$
    \item zero \\
\end{enumerate}
\item In a rectangular channel, the ratio of the velocity head to the flow depth for critical flow condition, is
\begin{enumerate}
    \item $\frac{1}{2}$
    \item $\frac{2}{3}$
    \item $\frac{3}{2}$
    \item 2 \\
\end{enumerate}
\item If the path of an irrigation canal is below the bed level of a natural stream, the type of cross-drainage structure provided is
\begin{enumerate}
    \item Aqueduct
    \item Level crossing
    \item Sluice gate
    \item Super passage \\
\end{enumerate}
\item A catchment may be idealised as a rectangle. There are three rain gauges located inside the catchment at arbitrary locations. The average precipitation over the catchment is estimated by two methods: (i) Arithmetic mean $\brak{P_A}$, and (ii) Thiessen polygon $\brak{P_T}$. Which one of the
following statements is correct ?
\begin{enumerate}
    \item $P_A$ is always smaller than $P_T$
    \item $P_A$ is always greater than $P_T$
    \item $P_A$ is always equal to $P_T$
    \item There is no definite relationship between $P_A$ and $P_T$ \\
\end{enumerate}
\item A retaining wall of height $H$ with smooth vertical backface supports a backfill inclined at an angle $\beta$ with the horizontal. The backfill consists of cohesionless soil having angle of internal friction $\phi$. If the active lateral thrust acting on the wall is $P_a$, which one of the following statements is TRUE ?
\begin{enumerate}
    \item $P_a$ acts at a height $\frac{H}{2}$ from the base of the wall and at an angle $\beta$ with the horizontal
    \item $P_a$ acts at a height $\frac{H}{2}$ from the base of the wall and at an angle $\phi$ with the horizontal
    \item $P_a$ acts at a height $\frac{H}{3}$ from the base of the wall and at an angle $\beta$ with the horizontal
    \item $P_a$ acts at a height $\frac{H}{3}$ from the base of the wall and at an angle $\phi$ with the horizontal \\
\end{enumerate}
			 \end{enumerate}
			 \end{document}
 
