\let\negmedspace\undefined
\let\negthickspace\undefined
\documentclass[journal]{IEEEtran}
\usepackage[a5paper, margin=10mm, onecolumn]{geometry}
%\usepackage{lmodern} % Ensure lmodern is loaded for pdflatex
\usepackage{tfrupee} % Include tfrupee package

\setlength{\headheight}{1cm} % Set the height of the header box
\setlength{\headsep}{0mm}     % Set the distance between the header box and the top of the text

\usepackage{gvv-book}
\usepackage{gvv}
\usepackage{cite}
\usepackage{amsmath,amssymb,amsfonts,amsthm}
\usepackage{algorithmic}
\usepackage{graphicx}
\usepackage{textcomp}
\usepackage{xcolor}
\usepackage{txfonts}
\usepackage{listings}
\usepackage{enumitem}
\usepackage{mathtools}
\usepackage{gensymb}
\usepackage{comment}
\usepackage[breaklinks=true]{hyperref}
\usepackage{tkz-euclide} 
\usepackage{listings}
% \usepackage{gvv}                                        
\def\inputGnumericTable{}                                 
\usepackage[latin1]{inputenc}                                
\usepackage{color}                                            
\usepackage{array}                                            
\usepackage{longtable}                                       
\usepackage{calc}                                             
\usepackage{multirow}                                         
\usepackage{hhline}                                           
\usepackage{ifthen}                                           
\usepackage{lscape}
\begin{document}

\bibliographystyle{IEEEtran}
\vspace{3cm}




\title{
%	\logo{
GATE - 2020 - ME

\large{EE1030 : Matrix Theory}

Indian Institute of Technology Hyderabad
%	}
}
\author{Satyanarayana Gajjarapu

AI24BTECH11009
}	





\maketitle




\bigskip

\renewcommand{\thefigure}{\theenumi}
\renewcommand{\thetable}{\theenumi}


\section{40 - 52}


\begin{enumerate}
\item The truss shown in the figure has four members of length $l$ and flexural rigidity $EI$, and one member of length $l\sqrt{2}$ and flexural rigidity $4EI$. The truss is loaded by a pair of forces of magnitude $P$, as shown in figure.
\begin{figure}[!ht]
\centering
\resizebox{0.5\textwidth}{!}{%
\begin{circuitikz}
\tikzstyle{every node}=[font=\normalsize]
\draw [short] (6.5,11.75) -- (6.5,8.25);
\draw [short] (4,10) -- (9,10);
\draw  (6.5,10) circle (1.25cm);
\node at (7,10.5) [circ] {};
\node [font=\normalsize] at (7.25,10.5) {$z$};
\node [font=\normalsize] at (6.75,12) {Im};
\node [font=\normalsize] at (9,9.75) {Re};
\node [font=\normalsize] at (8,11.75) {Unit circle};
\draw [->, >=Stealth] (7.5,11.5) -- (7.25,11.25);
\end{circuitikz}

}%
\end{figure}\\
The smallest value of $P$, at which any of the truss members will buckle is
\begin{enumerate}
    \item $\frac{\sqrt{2}\pi^2EI}{l^2}$
    \item $\frac{\pi^2EI}{l^2}$
    \item $\frac{2\pi^2EI}{l^2}$
    \item $\frac{\pi^2EI}{2l^2}$ \\
\end{enumerate}
\item A rigid mass-less rod of length $L$ is connected to a disc (pulley) of mass $m$ and radius $r = \frac{L}{4}$ through a friction-less revolute joint. The other end of that rod is attached to a wall through a friction-less hinge. A spring of stiffness $2k$ is attached to the rod at its mid-span. An inextensible rope passes over half the disc periphery and is securely tied to a spring of stiffness $k$ at point C as shown in the figure. There is no slip
between the rope and the pulley. The system is in static equilibrium in the configuration shown in the figure and the rope is always taut.
\pagebreak
\begin{figure}[!ht]
\centering
\resizebox{0.5\textwidth}{!}{%
\begin{circuitikz}
\tikzstyle{every node}=[font=\normalsize]
\draw [->, >=Stealth] (6.25,9.5) -- (6.25,11);
\draw [->, >=Stealth] (6.25,9.5) -- (14.5,9.5);
\draw [short] (6.25,9.25) -- (6.25,8.75);
\draw [short] (12.75,9.25) -- (12.75,8.75);
\draw [<->, >=Stealth] (6.25,9) -- (12.75,9);
\draw [line width=1.6pt, short] (6.25,9.5) -- (12.75,9.5);
\draw [->, >=Stealth] (2.5,10.75) -- (5,10.75);
\draw [->, >=Stealth] (2.5,10) -- (5,10);
\draw [->, >=Stealth] (2.5,9.25) -- (5,9.25);
\draw [->, >=Stealth] (2.5,8.5) -- (5,8.5);
\node [font=\normalsize] at (9.25,8.75) {0.25 m};
\node [font=\normalsize] at (14.75,9.5) {$x$};
\node [font=\normalsize] at (6.25,11.25) {$y$};
\node [font=\normalsize] at (3.5,11) {$U_{\infty}$};
\end{circuitikz}

}%
\end{figure}\\
Neglecting the influence of gravity, the natural frequency of the system for small amplitude vibration is 
\begin{enumerate}
    \item $\sqrt{\frac{3}{2}}\sqrt{\frac{k}{m}}$
    \item $\frac{3}{\sqrt{2}}\sqrt{\frac{k}{m}}$
    \item $\sqrt{3}\sqrt{\frac{k}{m}}$
    \item $\sqrt{\frac{k}{m}}$ \\
\end{enumerate}
\item A strip of thickness 40 mm is to be rolled to a thickness of 20 mm using a two-high mill having rolls of diameter 200 mm. Coefficient of friction and arc length in mm, respectively are
\begin{enumerate}
    \item 0.45 and 38.84
    \item 0.39 and 38.84
    \item 0.39 and 44.72
    \item 0.45 and 44.72 \\
\end{enumerate}
\item For an assembly line, the production rate was 4 pieces per hour and the average processing time was 60 minutes. The WIP inventory was calculated. Now, the production rate is kept the same, and the average processing time is brought down by 30 percent. As a result of this change in the processing time, the WIP inventory.
\begin{enumerate}
    \item decreases by 25\%
    \item increases by 25\%
    \item decreases by 30\%
    \item increases by 30\% \\
\end{enumerate}
\item A small metal bead (radius 0.5 mm), initially at 100\degree C, when placed in a stream of fluid at 20\degree C, attains a temperature of 28\degree C in 4.35 seconds. The density and specific heat of the metal are 8500 kg/$\text{m}^3$ and 400 J/kg$\cdot$K, respectively. If the bead is considered as lumped system, the convective heat transfer coefficient (in W/$\text{m}^2\cdot$K) between the metal bead and the fluid stream is
\begin{enumerate}
    \item 283.3
    \item 299.8
    \item 149.9
    \item 449.7 \\
\end{enumerate}
\item Consider two exponentially distributed random variables $X$ and $Y$, both having a mean of 0.50. Let $Z = X + Y$ and $r$ be the correlation coefficient between $X$ and $Y$. If the variance of $Z$ equals 0, then the
value of $r$ is $\_\_\_\_$ \brak{round\ off\ to\ 2\ decimal\ places}. \\
\item An analytic function of a complex variable $z = x + iy\ \brak{i = \sqrt{-1}}$ is defined as 
\begin{align*}
    f\brak{z} = x^2 - y^2 + i\psi\brak{x,y},
\end{align*}
where $\psi\brak{x,y}$ is a real function. The value of the imaginary part of $f\brak{z}$ at $z = \brak{1 + i}$ is $\_\_\_\_$ \brak{round\ off\ to\ 2\ decimal\ places}. \\
\item In a disc-type axial clutch, the friction contact takes places within an annular region with outer and inner diameters 250 mm and 50 mm, respectively. An axial force $F_1$ is needed to transmit a torque by a new clutch. However, to transmit the same torque, one needs an axial force $F_2$ when the clutch wears out. If contact pressure remains uniform during operation of a new clutch while the wear is assumed to be uniform for an old clutch, and the coefficient of friction does not change, then the ratio $\frac{F_1}{F_2}$ is
$\_\_\_\_$ \brak{round\ off\ to\ 2\ decimal\ places}. \\
\item A cam with translating flat-face follower is desired to have the follower motion 
\begin{align*}
    y\brak{\theta} = 4 \sbrak{2\pi\theta - \theta^2},\ 0 \leq \theta \leq 2\pi
\end{align*}
Contact stress considerations dictate that the radius of curvature of the cam profile should not be less than 40 mm anywhere. The minimum permissible base circle radius is $\_\_\_\_$ mm \brak{round\ off\ to\ one\ decimal\ place}. \\
\item A rectangular steel bar of length 500 mm, width 100 mm, and thickness 15 mm is cantilevered to a 200 mm steel channel using 4 bolts, as shown.
\pagebreak
\begin{figure}[!ht]
\centering
\resizebox{0.7\textwidth}{!}{%
\begin{circuitikz}
\tikzstyle{every node}=[font=\normalsize]
\draw [short] (4,10) -- (4,7.5);
\draw [line width=1.4pt, short] (4,7.5) -- (12.5,7.5);
\draw [short] (12.5,7.5) -- (12.5,10);
\draw  (5.75,9) rectangle (7.25,7.5);
\draw  (8.75,9) rectangle (10.25,7.5);
\draw (4,8.5) to[R] (5.75,8.5);
\draw (7.25,8.5) to[R] (8.75,8.5);
\draw (10.25,8.5) to[R] (12.5,8.5);
\draw [line width=1.6pt, short] (6.5,9) -- (6.5,10);
\draw [line width=1.6pt, short] (9.5,9) -- (9.5,10);
\draw [->, >=Stealth] (6.5,9.5) -- (7.25,9.5);
\draw [->, >=Stealth] (9.5,9.5) -- (10.25,9.5);
\draw [->, >=Stealth] (8,7) .. controls (9,7) and (9,7) .. (9.25,7.5) ;
\node [font=\normalsize] at (5,9) {$k$};
\node [font=\normalsize] at (8,9) {$k$};
\node [font=\normalsize] at (11.25,9) {$2k$};
\node [font=\normalsize] at (7,9.75) {$x_1\brak{t}$};
\node [font=\normalsize] at (10,9.75) {$x_2\brak{t}$};
\node [font=\normalsize] at (6.75,7) {Smooth surface};
\end{circuitikz}

}%
\end{figure}\\
For an external load of 10 kN applied at the tip of steel bar, the resultant shear load on the bolt at B, is $\_\_\_\_$ kN (round off to one decimal place). \\
\item The barrier shown between two water tanks of unit width (1 m) into the plane of the screen is modeled as a cantilever.
\begin{figure}[!ht]
\centering
\resizebox{0.5\textwidth}{!}{%
\begin{circuitikz}
\tikzstyle{every node}=[font=\normalsize]
\draw (9,10) to[short] (9.25,10);
\draw (9,9.5) to[short] (9.25,9.5);
\draw (9.25,10) node[ieeestd xor port, anchor=in 1, scale=0.89](port){} (port.out) to[short] (11,9.75);
\draw (5,9.5) node[ieeestd not port, anchor=in](port){} (port.out) to[short] (7,9.5);
\draw (port.in) to[short] (4.5,9.5);
\draw [short] (7,9.5) -- (9,9.5);
\draw [short] (3.75,10) -- (9.25,10);
\draw [short] (4.5,10) -- (4.5,9.5);
\node [font=\normalsize] at (11.5,9.75) {\textbf{Y}};
\node [font=\normalsize] at (3.5,10) {\textbf{X}};
\end{circuitikz}

}%
\end{figure}\\
Taking the density of water as 1000 kg/$\text{m}^3$, and the acceleration due to gravity as 10 m/$\text{s}^2$, the maximum absolute bending moment developed in the cantilever is $\_\_\_\_$ kN$\cdot$m \brak{round\ off\ to\ the\
nearest\ integer}. \\
\item The magnitude of reaction force at joint C of the hinge-beam shown in the figure is $\_\_\_\_$ kN \brak{round\ off\ to\ 2\ decimal\ places}.
\begin{figure}[!ht]
\centering
\resizebox{0.5\textwidth}{!}{%
\begin{circuitikz}
\tikzstyle{every node}=[font=\normalsize]
\draw [short] (3,10.75) -- (3,8);
\draw  (3,9.5) rectangle (10.5,9.25);
\draw  (7.5,9.5) rectangle (7.25,9.25);
\draw [short] (7.25,10.5) -- (10.5,10.5);
\draw [->, >=Stealth] (7.5,10.5) -- (7.5,9.5);
\draw [->, >=Stealth] (8,10.5) -- (8,9.5);
\draw [->, >=Stealth] (8.5,10.5) -- (8.5,9.5);
\draw [->, >=Stealth] (9,10.5) -- (9,9.5);
\draw [->, >=Stealth] (9.5,10.5) -- (9.5,9.5);
\draw [->, >=Stealth] (10,10.5) -- (10,9.5);
\draw [->, >=Stealth] (10.5,10.5) -- (10.5,9.5);
\draw [->, >=Stealth] (5.25,11.25) -- (5.25,9.5);
\draw [short] (5.25,9) -- (5.25,8.25);
\draw [short] (7.5,9) -- (7.5,8.25);
\draw [short] (10.5,9.5) -- (10.25,9);
\draw [short] (10.5,9.5) -- (10.75,9);
\draw [short] (10.25,9) -- (10.75,9);
\draw [short] (9.75,8.75) -- (11.25,8.75);
\node [font=\Large] at (10.25,9) {o};
\node [font=\Large] at (10.5,9) {o};
\node [font=\Large] at (10.75,9) {o};
\draw [short] (10,8.75) -- (9.75,8.5);
\draw [short] (10.25,8.75) -- (10,8.5);
\draw [short] (10.5,8.75) -- (10.25,8.5);
\draw [short] (10.75,8.75) -- (10.5,8.5);
\draw [short] (11,8.75) -- (10.75,8.5);
\draw [<->, >=Stealth] (3,8.5) -- (5.25,8.5);
\draw [<->, >=Stealth] (5.25,8.5) -- (7.5,8.5);
\draw [<->, >=Stealth] (7.5,8.5) -- (10.5,8.5);
\node [font=\normalsize] at (5.25,11.5) {50 kN};
\node [font=\normalsize] at (9,10.75) {10 kN/m};
\node [font=\normalsize] at (3.25,9.75) {A};
\node [font=\normalsize] at (7,9.75) {B};
\node [font=\normalsize] at (10.75,9.5) {C};
\node [font=\normalsize] at (4,8.25) {3 m};
\node [font=\normalsize] at (6.25,8.25) {3 m};
\node [font=\normalsize] at (8.75,8.25) {4 m};
\draw [short] (3,10.75) -- (2.75,10.5);
\draw [short] (3,10.5) -- (2.75,10.25);
\draw [short] (3,10.25) -- (2.75,10);
\draw [short] (3,10) -- (2.75,9.75);
\draw [short] (3,9.75) -- (2.75,9.5);
\draw [short] (3,9.5) -- (2.75,9.25);
\draw [short] (3,9.25) -- (5,9.25);
\draw [short] (3,9.25) -- (2.75,9);
\draw [short] (3,9) -- (2.75,8.75);
\draw [short] (3,8.75) -- (2.75,8.5);
\draw [short] (3,8.5) -- (2.75,8.25);
\draw [short] (3,8.25) -- (2.75,8);
\draw [->, >=Stealth] (9,7.25) .. controls (8.25,10.75) and (8.25,7) .. (7.5,9.25) ;
\node [font=\normalsize] at (9,7) {Hinge};
\end{circuitikz}

}%
\end{figure}\\
\item A slot of 25 mm $\times$ 25 mm is to be milled in a workpiece of 300 mm length using a side and face milling cutter of diameter 100 mm, width 25 mm and having 20 teeth.\\\\
For a depth of cut 5 mm, feed per tooth 0.1 mm, cutting speed 35 m/min and approach and over travel distance of 5 mm each, the time required for milling the slot is $\_\_\_\_$ minutes \brak{round\ off\ to\ one\ decimal\ place}. \\
			 \end{enumerate}
			 \end{document}
 
