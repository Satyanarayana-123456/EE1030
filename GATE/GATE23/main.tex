\let\negmedspace\undefined
\let\negthickspace\undefined
\documentclass[journal]{IEEEtran}
\usepackage[a5paper, margin=10mm, onecolumn]{geometry}
%\usepackage{lmodern} % Ensure lmodern is loaded for pdflatex
\usepackage{tfrupee} % Include tfrupee package

\setlength{\headheight}{1cm} % Set the height of the header box
\setlength{\headsep}{0mm}     % Set the distance between the header box and the top of the text

\usepackage{gvv-book}
\usepackage{gvv}
\usepackage{cite}
\usepackage{amsmath,amssymb,amsfonts,amsthm}
\usepackage{algorithmic}
\usepackage{graphicx}
\usepackage{textcomp}
\usepackage{xcolor}
\usepackage{txfonts}
\usepackage{listings}
\usepackage{enumitem}
\usepackage{mathtools}
\usepackage{gensymb}
\usepackage{comment}
\usepackage[breaklinks=true]{hyperref}
\usepackage{tkz-euclide} 
\usepackage{listings}
% \usepackage{gvv}                                        
\def\inputGnumericTable{}                                 
\usepackage[latin1]{inputenc}                                
\usepackage{color}                                            
\usepackage{array}                                            
\usepackage{longtable}                                       
\usepackage{calc}                                             
\usepackage{multirow}                                         
\usepackage{hhline}                                           
\usepackage{ifthen}                                           
\usepackage{lscape}
\begin{document}

\bibliographystyle{IEEEtran}
\vspace{3cm}




\title{
%	\logo{
GATE - 2020 - XE

\large{EE1030 : Matrix Theory}

Indian Institute of Technology Hyderabad
%	}
}
\author{Satyanarayana Gajjarapu

AI24BTECH11009
}	





\maketitle




\bigskip

\renewcommand{\thefigure}{\theenumi}
\renewcommand{\thetable}{\theenumi}


\section{40 - 52}


\begin{enumerate}
\item A venturimeter with 75 mm diameter throat is placed in a 150 mm diameter pipeline carrying water at 25\degree C. The pressure drop between the upstream tap and the venturi throat is 40 kPa. (Density of water = 1000 kg/$\text{m}^3$). \\\\
The flow rate is $\_\_\_\_$ $\text{m}^3$/s (rounded off to three decimal places). \\
\item A water jet with velocity $\overrightarrow{V}_{\text{jet}}$ impinges normal to a moving flat plate with velocity $\overrightarrow{V}_{\text{plate}}$ such that the jet splits equally into two halves as shown in Figure. The jet cross-sectional area is 2 $\text{cm}^2$, $\overrightarrow{V}_{\text{jet}}$ is 20 m/s and $\overrightarrow{V}_{\text{plate}}$ is 10 m/s and density of water is 1000 kg/$\text{m}^3$. Consider steady flow and neglect weight of the jet, weight of the plate and frictional losses. \\\\
The absolute value of the force required to keep the plate moving at constant velocity $\overrightarrow{V}_{\text{plate}}$ is $\_\_\_\_$ N.
\begin{figure}[!ht]
\centering
\resizebox{0.5\textwidth}{!}{%
\begin{circuitikz}
\tikzstyle{every node}=[font=\normalsize]
\draw [short] (2.5,6.75) -- (16.75,7);
\draw [short] (4.25,8) -- (4.25,6.75);
\draw [short] (5.75,9.25) -- (5.75,6.75);
\draw [short] (7.25,10.25) -- (7.25,6.75);
\draw [short] (2.5,6.75) -- (9,11.75);
\draw [short] (9,11.75) -- (16.5,7);
\draw [short] (10.75,10.5) -- (10.75,7);
\draw [short] (12.25,9.5) -- (12.25,7);
\draw [short] (14,8.5) -- (14,7);
\draw [short] (4.25,6.75) -- (5.75,9.25);
\draw [short] (5.75,6.75) -- (7.25,10.25);
\draw [short] (7.25,7) -- (9,11.75);
\draw [short] (9,11.75) -- (10.75,7);
\draw [short] (10.75,10.5) -- (12.25,7);
\draw [short] (12.25,9.5) -- (14,7);
\draw [->, >=Stealth] (5.75,10.25) -- (5.75,9.5);
\draw [->, >=Stealth] (7.25,11.25) -- (7.25,10.5);
\draw [->, >=Stealth] (10.75,11.5) -- (10.75,10.75);
\draw [->, >=Stealth] (12.25,10.75) -- (12.25,10);
\draw [short] (2.25,5.5) -- (16.75,5.75);
\draw [short] (2.5,5.75) -- (2.5,5.25);
\draw [short] (4.25,5.75) -- (4.25,5.25);
\draw [short] (5.75,5.75) -- (5.75,5.25);
\draw [short] (7.25,5.75) -- (7.25,5.25);
\draw [short] (10.75,5.75) -- (10.75,5.25);
\draw [short] (12.5,6) -- (12.5,5.25);
\draw [short] (14,6) -- (14,5.5);
\draw [short] (16.5,6) -- (16.5,5.5);
\draw [short] (2.5,6.75) -- (2.25,6.5);
\draw [short] (2.5,6.75) -- (2.75,6.5);
\draw [short] (2.25,6.5) -- (2.75,6.5);
\draw [short] (2,6.5) -- (3.25,6.5);
\draw [short] (2,6.5) -- (2.25,6.25);
\draw [short] (2.25,6.5) -- (2.5,6.25);
\draw [short] (2.5,6.5) -- (2.75,6.25);
\draw [short] (2.75,6.5) -- (3,6.25);
\draw [short] (3,6.5) -- (3.25,6.25);
\draw [short] (16.5,7) -- (16.25,6.75);
\draw [short] (16.5,7) -- (16.75,6.75);
\draw [short] (16,6.75) -- (17,6.75);
\draw  (16,6.5) circle (0.25cm);
\draw  (17,6.5) circle (0.25cm);
\draw [short] (15.25,6.25) -- (17.25,6.25);
\draw [short] (15.25,6.25) -- (15.5,6);
\draw [short] (15.5,6.25) -- (15.75,6);
\draw [short] (16,6.25) -- (16.25,6);
\draw [short] (15.75,6.25) -- (16,6);
\draw [short] (16.25,6.25) -- (16.5,6);
\draw [short] (16.5,6.25) -- (16.75,6);
\draw [short] (16.75,6.25) -- (17,6);
\draw [short] (17,6.25) -- (17.25,6);
\draw [short] (17.75,12.25) -- (17.75,7);
\draw [short] (17.5,7.25) -- (18,7.25);
\draw [short] (17.5,11.75) -- (18,11.75);
\draw [short] (17.5,10.75) -- (18,10.75);
\draw [short] (17.5,9.75) -- (18,9.75);
\draw [short] (17.5,8.5) -- (18,8.5);
\node [font=\normalsize] at (5.75,10.5) {20 kN};
\node [font=\normalsize] at (7.25,11.5) {20 kN};
\node [font=\normalsize] at (10.75,11.75) {20 kN};
\node [font=\normalsize] at (12.25,11) {20 kN};
\node [font=\normalsize] at (9,12) {L};
\node [font=\normalsize] at (3.25,5.25) {2 m};
\node [font=\normalsize] at (5,5.25) {2 m};
\node [font=\normalsize] at (6.5,5.25) {2 m};
\node [font=\normalsize] at (9,5.25) {2 m};
\node [font=\normalsize] at (11.5,5.25) {2 m};
\node [font=\normalsize] at (13.25,5.25) {2 m};
\node [font=\normalsize] at (15.25,5.25) {2 m};
\node [font=\normalsize] at (18.25,11.25) {1 m};
\node [font=\normalsize] at (18.25,10.25) {1 m};
\node [font=\normalsize] at (18.25,9) {1 m};
\node [font=\normalsize] at (18.25,8) {1 m};
\node [font=\normalsize] at (2.5,7) {E};
\node [font=\normalsize] at (16.75,7.25) {T};
\node [font=\normalsize] at (4.25,6.5) {F};
\node [font=\normalsize] at (4.25,8.25) {G};
\node [font=\normalsize] at (5.5,9.25) {I};
\node [font=\normalsize] at (5.75,6.5) {H};
\node [font=\normalsize] at (7.5,6.5) {J};
\node [font=\normalsize] at (7.5,10.75) {K};
\node [font=\normalsize] at (10.75,6.75) {M};
\node [font=\normalsize] at (11,10.75) {N};
\node [font=\normalsize] at (12.25,6.75) {O};
\node [font=\normalsize] at (12.5,10) {P};
\node [font=\normalsize] at (14,6.75) {R};
\node [font=\normalsize] at (14.25,8.75) {S};
\end{circuitikz}

}%
\end{figure}\\
\item In an inverted manometer (as shown in the Figure), the pressure difference, $p_B - p_A$ is 100 kPa.\\\\
Use specific gravity of oil as 0.8, density of water as 1000 kg/$\text{m}^3$, density of mercury as 13600 kg/$\text{m}^3$ and acceleration due to gravity as 10 m/$\text{s}^2$. \\\\
The height of the water column, $H$ is $\_\_\_\_$ cm. (rounded off to one decimal place). \\
\begin{figure}[!ht]
\centering
\resizebox{0.5\textwidth}{!}{%
\begin{circuitikz}
\tikzstyle{every node}=[font=\normalsize]
\draw (5.75,10.75) to[R] (8.5,10.75);
\draw (5.75,11.75) to[R] (8.5,11.75);
\draw (5.75,11.75) to[short] (5.75,10.75);
\draw (8.5,11.75) to[short] (8.5,10.75);
\draw (8.5,11.25) to[short] (9.75,11.25);
\draw (5.75,11.25) to[short] (4.75,11.25);
\draw (9.75,11.25) to[R] (9.75,8.75);
\draw (4.75,11.25) to[battery1] (4.75,8.75);
\draw [->, >=Stealth] (6.5,11.5) -- (7.75,12);
\draw [short] (9.75,8.75) -- (4.75,8.75);
\node [font=\normalsize] at (5,10.25) {$+$};
\node [font=\normalsize] at (4,10) {10 V};
\node [font=\normalsize] at (7,12.5) {R};
\node [font=\normalsize] at (7,10.25) {6 $\Omega$};
\node [font=\normalsize] at (8.75,10) {3 $\Omega$};
\draw [ dashed] (9.25,10.75) rectangle  (10.25,9.25);
\node at (5.75,11.25) [circ] {};
\node at (8.5,11.25) [circ] {};
\node [font=\normalsize] at (10.75,10) {Load};
\end{circuitikz}

}%
\end{figure}
\item An incompressible, steady flow with a uniform velocity condition at the inlet between parallel plates is shown in Figure. The flow develops into a parabolic laminar profile with $u = ay\brak{y_0 - y}$ at the downstream end, where '$a$' is a constant. Assume unit depth of the plate. For $U_0$ = 7.5 cm/s, $y_0$ = 3 cm and the fluid with density, $\rho$ = 800 kg/$\text{m}^3$. \\\\
The value of $a$ is $\_\_\_\_$. \\
\begin{figure}[!ht]
\centering
\resizebox{0.7\textwidth}{!}{%
\begin{circuitikz}
\tikzstyle{every node}=[font=\normalsize]
\draw [short] (5,10.5) -- (5,9.25);
\draw [short] (5,9.25) -- (11.5,9.25);
\draw [short] (11.5,10.5) -- (11.5,9.25);
\draw [short] (5,7.5) -- (11.5,7.5);
\draw [short] (11.5,7.5) -- (11.5,6.25);
\draw [short] (5,7.5) -- (5,6.25);
\draw [->, >=Stealth] (5,9) -- (6.25,9);
\draw [->, >=Stealth] (5,8.75) -- (6.25,8.75);
\draw [->, >=Stealth] (5,8.25) -- (6.25,8.25);
\draw [->, >=Stealth] (5,7.75) -- (6.25,7.75);
\draw [->, >=Stealth] (5,8) -- (6.25,8);
\draw [->, >=Stealth] (5,8.5) -- (6.25,8.5);
\draw [short] (11.5,9.25) .. controls (14,8.5) and (12.75,8) .. (11.5,7.5);
\draw [->, >=Stealth] (1.5,8) -- (1.5,9.75);
\draw [->, >=Stealth] (1.5,8) -- (3,8);
\draw [->, >=Stealth] (11.5,9) -- (12,9);
\draw [->, >=Stealth] (11.5,8.75) -- (12.5,8.75);
\draw [->, >=Stealth] (11.5,8.5) -- (12.75,8.5);
\draw [->, >=Stealth] (11.5,8.25) -- (12.75,8.25);
\draw [->, >=Stealth] (11.5,8) -- (12.5,8);
\draw [->, >=Stealth] (11.5,7.75) -- (12.25,7.75);
\draw [line width=0.2pt, short] (11.5,9.25) -- (11.5,7.5);
\draw [short] (5,9.25) -- (6.5,10.5);
\draw [short] (6,9.25) -- (7.5,10.25);
\draw [short] (7.25,9.25) -- (8.5,10.25);
\draw [short] (8.5,9.25) -- (9.75,10.25);
\draw [short] (9.75,9.25) -- (10.75,10.25);
\draw [short] (10.75,9.25) -- (11.5,10);
\draw [short] (5.5,9.25) -- (7,10.5);
\draw [short] (6.75,9.25) -- (8,10.25);
\draw [short] (8,9.25) -- (9.25,10.5);
\draw [short] (9.25,9.25) -- (10.25,10.25);
\draw [short] (10.25,9.25) -- (11.25,10.25);
\draw [short] (5,7) -- (5.75,7.5);
\draw [short] (5,6.5) -- (6.75,7.5);
\draw [short] (6,6.75) -- (7.5,7.5);
\draw [short] (7.25,6.5) -- (8.75,7.5);
\draw [short] (8.25,6.5) -- (9.5,7.5);
\draw [short] (9,6.25) -- (10,7.5);
\draw [short] (10.75,6.5) -- (11.5,7.5);
\draw [short] (10,6.5) -- (10.75,7.5);
\draw [short] (6.5,6.5) -- (8,7.5);
\node [font=\normalsize] at (12.25,9.5) {$y = y_0$};
\node [font=\normalsize] at (12.25,7.25) {$y = 0$};
\node [font=\normalsize] at (4.5,8.5) {$U_0$};
\node [font=\normalsize] at (3,7.75) {$x$};
\node [font=\normalsize] at (1.75,10) {$y$};
\end{circuitikz}

}%
\end{figure}
\item A Pb-Sn sample of eutectic composition, containing $\alpha$- and $\beta$-phases, is examined in a scanning electron microscope. The $\alpha$-phase contains $\sim$ 97 wt\% Pb (atomic number 82) while $\beta$-phase contains $\sim$ 99 wt\% Sn (atomic number 50). The ratio of number of backscattered electrons escaping from $\alpha$-phase to that from $\beta$-phase would be:
\begin{enumerate}
    \item Less than 1
    \item Equal to 1
    \item Greater than 1
    \item Equal to 0 \\
\end{enumerate}
\item Smallest or minimum feature size that can be theoretically resolved in an optical microscope does NOT depend on:
\begin{enumerate}
    \item Refractive index of the medium between the lens and the focal point
    \item Intensity of radiation
    \item Wavelength of radiation
    \item Numerical aperture of the objective lens \\ 
\end{enumerate}
\item Following diagram shows a square 2-D lattice with a hexagonal motif (dark colored). The rotational symmetry element that must be present in the system is:
\begin{figure}[!ht]
\centering
\resizebox{0.3\textwidth}{!}{%
\begin{tabular}[12pt]{ |c| c|}
    \hline
    \textbf{Triangle} & \textbf{Interior Angles}\\ 
    \hline
    P & 85\degree, 50\degree, 45\degree \\
    \hline 
    Q & 100\degree, 55\degree, 25\degree \\
    \hline
    R & 100\degree, 45\degree, 35\degree \\
    \hline
    S & 130\degree, 30\degree, 20\degree \\
    \hline
    \end{tabular}


}%
\end{figure}
\begin{enumerate}
    \item Six-fold rotation
    \item Two-fold rotation
    \item Three-fold rotation
    \item Four-fold rotation \\
\end{enumerate}
\item Density of states, $D\brak{E}$, in a three dimensional solid varies with energy \brak{E} as
\begin{enumerate}
    \item $E^{\frac{1}{2}}$
    \item $E^0$
    \item $E^{-\frac{1}{2}}$
    \item $E^{\frac{3}{2}}$ \\
\end{enumerate}
\item The variation of molar volume (Vm) of a liquid showing glass transition temperature (Tg) while cooling from its melting temperature (Tm) is depicted by:
\pagebreak
\begin{figure}[!ht]
\centering
\resizebox{0.7\textwidth}{!}{%
\begin{tabular}[12pt]{ |c| c|}
    \hline
    \textbf{Type of Stress} & \textbf{Location}\\ 
    \hline
    P. Load & 1. Corner \\
    \hline 
    Q. Temperature & 2. Edge \\
    \hline
     & 3. Interior \\
    \hline
    \end{tabular}

}%
\end{figure}
\begin{enumerate}
    \item I
    \item II
    \item III
    \item IV \\
\end{enumerate}
\item Find the correct match between polymer name in Column I and the monomer type in Column II.
\begin{table}[h!]
  \centering
  \begin{circuitikz}
\tikzstyle{every node}=[font=\normalsize]
\draw [short] (4,10) -- (4,7.5);
\draw [line width=1.4pt, short] (4,7.5) -- (12.5,7.5);
\draw [short] (12.5,7.5) -- (12.5,10);
\draw  (5.75,9) rectangle (7.25,7.5);
\draw  (8.75,9) rectangle (10.25,7.5);
\draw (4,8.5) to[R] (5.75,8.5);
\draw (7.25,8.5) to[R] (8.75,8.5);
\draw (10.25,8.5) to[R] (12.5,8.5);
\draw [line width=1.6pt, short] (6.5,9) -- (6.5,10);
\draw [line width=1.6pt, short] (9.5,9) -- (9.5,10);
\draw [->, >=Stealth] (6.5,9.5) -- (7.25,9.5);
\draw [->, >=Stealth] (9.5,9.5) -- (10.25,9.5);
\draw [->, >=Stealth] (8,7) .. controls (9,7) and (9,7) .. (9.25,7.5) ;
\node [font=\normalsize] at (5,9) {$k$};
\node [font=\normalsize] at (8,9) {$k$};
\node [font=\normalsize] at (11.25,9) {$2k$};
\node [font=\normalsize] at (7,9.75) {$x_1\brak{t}$};
\node [font=\normalsize] at (10,9.75) {$x_2\brak{t}$};
\node [font=\normalsize] at (6.75,7) {Smooth surface};
\end{circuitikz}

\end{table}
\begin{enumerate}
    \item I-P, II-S, III-R, IV-Q
    \item I-R, II-Q, III-S, IV-P
    \item I-S, II-P, III-Q, IV-R
    \item I-S, II-R, III-Q, IV-P \\
\end{enumerate}
\item A ceramic has a fracture toughness $\brak{K_{Ic}}$ of 1 MPa$\cdot\text{m}^\frac{1}{2}$. If this ceramic is to be exposed to a maximum stress $\brak{\sigma}$ of 200 MPa, the maximum value of half crack length '$a$' (in micrometer, $\mu$m, below which the material does not fail, is $\_\_\_\_$ $\mu$m \brak{round\ off\ to\ one\ decimal\ place}. Loading condition for the sample is shown in the schematic. Assume geometrical factor f = 1.2. \\
\begin{figure}[!ht]
\centering
\resizebox{0.2\textwidth}{!}{%
\begin{tabular}[12pt]{ |c| c|}
    \hline
    \textbf{Processes} & \textbf{Characteristics / Applications} \\ 
    \hline
    P. Friction Welding & 1. Non-consumable electrode  \\
    \hline 
    Q. Gas Metal Arc Welding & 2. Joining of thick plates \\
    \hline
    R. Tungsten Inert Gas Welding & 3. Consumable electrode wire \\
    \hline
    S. Electroslag Welding & 4. Joining of cylindrical dissimilar materials \\
    \hline
    \end{tabular}

}%
\end{figure}
\item A ceramic material is periodically heated and cooled between 25\degree C and a higher temperature, $T_f$. During thermal cycling, the material remains dimensionally constrained. The material can withstand a maximum compressive stress of 200 MPa without failure. Material's coefficient of thermal expansion is 7.5$\times 10^{-6} \degree \text{C}^{-1}$ and modulus of elasticity \brak{E} is 200 GPa. The lowest value of $T_f$ (in \degree C) at which material will fail is $\_\_\_\_$ \degree C \brak{round-off\ to\ the\ nearest\ integer}. Assume that there is no plastic deformation during thermal cycling. \\
\item During homogeneous solidification of a liquid metal, the radius of critical nucleus (in nanometer, nm) at a temperature $T_s$ which is below the melting point $\brak{T_m}$, is $\_\_\_\_$ nm \brak{round-off\ to\ one\ decimal\ place}. Given that $\gamma_{sl}$ (solid-liquid interfacial energy) is 0.18 J$\cdot\text{m}^{-2}$ and $\Delta G_v$ (change in volume free energy upon transformation from liquid to solid) at $T_s$ is 0.18$\times 10^{8}$ J$\cdot\text{m}^{-3}$. \\
			 \end{enumerate}
			 \end{document}
 
