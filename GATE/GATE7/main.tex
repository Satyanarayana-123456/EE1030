\let\negmedspace\undefined
\let\negthickspace\undefined
\documentclass[journal]{IEEEtran}
\usepackage[a5paper, margin=10mm, onecolumn]{geometry}
%\usepackage{lmodern} % Ensure lmodern is loaded for pdflatex
\usepackage{tfrupee} % Include tfrupee package

\setlength{\headheight}{1cm} % Set the height of the header box
\setlength{\headsep}{0mm}     % Set the distance between the header box and the top of the text

\usepackage{gvv-book}
\usepackage{gvv}
\usepackage{cite}
\usepackage{amsmath,amssymb,amsfonts,amsthm}
\usepackage{algorithmic}
\usepackage{graphicx}
\usepackage{textcomp}
\usepackage{xcolor}
\usepackage{txfonts}
\usepackage{listings}
\usepackage{enumitem}
\usepackage{mathtools}
\usepackage{gensymb}
\usepackage{comment}
\usepackage[breaklinks=true]{hyperref}
\usepackage{tkz-euclide} 
\usepackage{listings}
% \usepackage{gvv}                                        
\def\inputGnumericTable{}                                 
\usepackage[latin1]{inputenc}                                
\usepackage{color}                                            
\usepackage{array}                                            
\usepackage{longtable}                                       
\usepackage{calc}                                             
\usepackage{multirow}                                         
\usepackage{hhline}                                           
\usepackage{ifthen}                                           
\usepackage{lscape}
\begin{document}

\bibliographystyle{IEEEtran}
\vspace{3cm}




\title{
%	\logo{
GATE - 2011 - EE

\large{EE1030 : Matrix Theory}

Indian Institute of Technology Hyderabad
%	}
}
\author{Satyanarayana Gajjarapu

AI24BTECH11009
}	





\maketitle




\bigskip

\renewcommand{\thefigure}{\theenumi}
\renewcommand{\thetable}{\theenumi}


\section{14 - 26}


\begin{enumerate}
\item A point $z$ has been plotted in the complex plane, as shown in figure below. 
\begin{figure}[!ht]
\centering
\resizebox{0.5\textwidth}{!}{%
\begin{circuitikz}
\tikzstyle{every node}=[font=\normalsize]
\draw [short] (6.5,11.75) -- (6.5,8.25);
\draw [short] (4,10) -- (9,10);
\draw  (6.5,10) circle (1.25cm);
\node at (7,10.5) [circ] {};
\node [font=\normalsize] at (7.25,10.5) {$z$};
\node [font=\normalsize] at (6.75,12) {Im};
\node [font=\normalsize] at (9,9.75) {Re};
\node [font=\normalsize] at (8,11.75) {Unit circle};
\draw [->, >=Stealth] (7.5,11.5) -- (7.25,11.25);
\end{circuitikz}

}%
\end{figure}\\
The plot of the complex number $y = \frac{1}{z}$ is
    \begin{enumerate}
        \item 
       \resizebox{0.25\textwidth}{!}{%
\begin{circuitikz}
\tikzstyle{every node}=[font=\normalsize]
\draw [short] (6.5,11.75) -- (6.5,8.25);
\draw [short] (4,10) -- (9,10);
\draw  (6.5,10) circle (1.25cm);
\node [font=\normalsize] at (6.75,12) {Im};
\node [font=\normalsize] at (9,9.75) {Re};
\node [font=\normalsize] at (8,11.75) {Unit circle};
\draw [->, >=Stealth] (7.5,11.5) -- (7.25,11.25);
\node at (7,9.5) [circ] {};
\node [font=\normalsize] at (6.75,9.25) {$y$};
\end{circuitikz}
}%
        \item 
       \resizebox{0.25\textwidth}{!}{%
\begin{circuitikz}
\tikzstyle{every node}=[font=\normalsize]
\draw [short] (6.5,11.75) -- (6.5,8.25);
\draw [short] (4,10) -- (9,10);
\draw  (6.5,10) circle (1.25cm);
\node [font=\normalsize] at (6.75,12) {Im};
\node [font=\normalsize] at (9,9.75) {Re};
\node [font=\normalsize] at (8,11.75) {Unit circle};
\draw [->, >=Stealth] (7.5,11.5) -- (7.25,11.25);
\node at (4.75,8.5) [circ] {};
\node [font=\normalsize] at (4.5,8.5) {$y$};
\end{circuitikz}
}%
        \item  
\resizebox{0.25\textwidth}{!}{%
\begin{circuitikz}
\tikzstyle{every node}=[font=\normalsize]
\draw [short] (6.5,11.75) -- (6.5,8.25);
\draw [short] (4,10) -- (9,10);
\draw  (6.5,10) circle (1.25cm);
\node [font=\normalsize] at (6.75,12) {Im};
\node [font=\normalsize] at (9,9.75) {Re};
\node [font=\normalsize] at (8,11.75) {Unit circle};
\draw [->, >=Stealth] (7.5,11.5) -- (7.25,11.25);
\node at (6,10.5) [circ] {};
\node [font=\normalsize] at (5.75,10.5) {$y$};
\end{circuitikz}
}%
        \item 
\resizebox{0.25\textwidth}{!}{%
\begin{circuitikz}
\tikzstyle{every node}=[font=\normalsize]
\draw [short] (6.5,11.75) -- (6.5,8.25);
\draw [short] (4,10) -- (9,10);
\draw  (6.5,10) circle (1.25cm);
\node [font=\normalsize] at (6.75,12) {Im};
\node [font=\normalsize] at (9,9.75) {Re};
\node [font=\normalsize] at (8,11.75) {Unit circle};
\draw [->, >=Stealth] (7.5,11.5) -- (7.25,11.25);
\node at (7.75,8.5) [circ] {};
\node [font=\normalsize] at (8,8.25) {$y$};
\end{circuitikz}
}%
    \end{enumerate}
\item The voltage applied to a circuit is $100\sqrt{2}\cos\brak{100 \pi t}$ volts and the circuit draws a current of $10\sqrt{2}\sin\brak{100 \pi t + \frac{\pi}{4}}$ amperes. Taking the voltage as the reference phasor, the phasor representation of the current in amperes is 
\begin{enumerate}
    \item $10\sqrt{2}\angle{-\frac{\pi}{4}}$
    \item $10\angle{-\frac{\pi}{4}}$
    \item $10\angle{+\frac{\pi}{4}}$
    \item $10\sqrt{2}\angle{+\frac{\pi}{4}}$ \\
\end{enumerate}
\item In the circuit given below, the value of R required for the transfer of maximum power to the load having a resistance of 3 $\Omega$ is 
\begin{figure}[!ht]
\centering
\resizebox{0.5\textwidth}{!}{%
\begin{circuitikz}
\tikzstyle{every node}=[font=\normalsize]
\draw (5.75,10.75) to[R] (8.5,10.75);
\draw (5.75,11.75) to[R] (8.5,11.75);
\draw (5.75,11.75) to[short] (5.75,10.75);
\draw (8.5,11.75) to[short] (8.5,10.75);
\draw (8.5,11.25) to[short] (9.75,11.25);
\draw (5.75,11.25) to[short] (4.75,11.25);
\draw (9.75,11.25) to[R] (9.75,8.75);
\draw (4.75,11.25) to[battery1] (4.75,8.75);
\draw [->, >=Stealth] (6.5,11.5) -- (7.75,12);
\draw [short] (9.75,8.75) -- (4.75,8.75);
\node [font=\normalsize] at (5,10.25) {$+$};
\node [font=\normalsize] at (4,10) {10 V};
\node [font=\normalsize] at (7,12.5) {R};
\node [font=\normalsize] at (7,10.25) {6 $\Omega$};
\node [font=\normalsize] at (8.75,10) {3 $\Omega$};
\draw [ dashed] (9.25,10.75) rectangle  (10.25,9.25);
\node at (5.75,11.25) [circ] {};
\node at (8.5,11.25) [circ] {};
\node [font=\normalsize] at (10.75,10) {Load};
\end{circuitikz}

}%
\end{figure}
\begin{enumerate}
    \item zero
    \item 3 $\Omega$
    \item 6 $\Omega$
    \item infinity \\
\end{enumerate}
\item Given two continuous time signals $x\brak{t} = e^{-t}$ and $y\brak{t} = e^{-2t}$ which exist for $t > 0$, the convolution $z\brak{t} = x\brak{t} * y\brak{t}$ is 
 \begin{enumerate}
     \item $e^{-t} - e^{-2t}$
     \item $e^{-3t}$
     \item $e^{+t}$
     \item $e^{-t} + e^{-2t}$ \\
 \end{enumerate}
\item A single phase air core transformer, fed from a rated sinusoidal supply, is operating at no load. The steady state magnetizing current drawn by the
transformer from the supply will have the waveform 
\begin{enumerate}
    \item  
\resizebox{0.3\textwidth}{!}{%
\begin{circuitikz}
\tikzstyle{every node}=[font=\normalsize]
\draw [short] (5.5,11) -- (5.5,7);
\draw [short] (4.75,9) -- (10.75,9);
\draw [short] (5.5,9) .. controls (6.25,9.5) and (6.5,9.5) .. (6.25,10);
\draw [short] (6.25,10) .. controls (7.25,11.25) and (7,9.75) .. (7.75,9);
\draw [short] (7.75,9) .. controls (8.75,7) and (8.5,8.75) .. (9,8.5);
\draw [short] (9,8.5) .. controls (9.5,8.75) and (9.5,8.75) .. (9.5,9);
\node [font=\normalsize] at (11,8.75) {t};
\node [font=\normalsize] at (5.25,11) {i};
\node [font=\normalsize] at (10.5,8.75) {$\rightarrow$};
\node [font=\normalsize] at (5.25,10.5) {$\uparrow$};
\end{circuitikz}
}%
    \item 
    \resizebox{0.3\textwidth}{!}{%
\begin{circuitikz}
\tikzstyle{every node}=[font=\normalsize]
\draw [short] (5.5,11) -- (5.5,7);
\draw [short] (4.75,9) -- (10.75,9);
\draw [short] (5.5,9) .. controls (6.25,9.5) and (6.5,9.5) .. (6.25,10);
\draw [short] (6.25,10) .. controls (7.75,11) and (7.25,9.75) .. (7.75,9);
\node [font=\normalsize] at (11,8.75) {t};
\node [font=\normalsize] at (5.25,11) {i};
\node [font=\normalsize] at (10.5,8.75) {$\rightarrow$};
\node [font=\normalsize] at (5.25,10.5) {$\uparrow$};
\draw [short] (7.75,9) .. controls (8,8.25) and (8.25,8.75) .. (8.75,8.5);
\draw [short] (8.75,8.5) .. controls (9.75,7.5) and (9.5,8.75) .. (10,9);
\end{circuitikz}
}%
    \item 
   \resizebox{0.3\textwidth}{!}{%
\begin{circuitikz}
\tikzstyle{every node}=[font=\normalsize]
\draw [short] (5.5,11) -- (5.5,7);
\draw [short] (4.75,9) -- (10.75,9);
\node [font=\normalsize] at (11,8.75) {t};
\node [font=\normalsize] at (5.25,11) {i};
\node [font=\normalsize] at (10.5,8.75) {$\rightarrow$};
\node [font=\normalsize] at (5.25,10.5) {$\uparrow$};
\draw [short] (5.5,9) .. controls (6.5,11.25) and (6.75,9.75) .. (7.25,9);
\draw [short] (7.25,9) .. controls (8.25,6.5) and (8.25,8) .. (8.75,9);
\end{circuitikz}
}%
    \item  
\resizebox{0.3\textwidth}{!}{%
\begin{circuitikz}
\tikzstyle{every node}=[font=\normalsize]
\draw [short] (5.5,11) -- (5.5,7);
\draw [short] (4.75,9) -- (10.75,9);
\node [font=\normalsize] at (11,8.75) {t};
\node [font=\normalsize] at (5.25,11) {i};
\node [font=\normalsize] at (10.5,8.75) {$\rightarrow$};
\node [font=\normalsize] at (5.25,10.5) {$\uparrow$};
\draw [short] (5.5,9) .. controls (7,9.5) and (5.75,10.5) .. (7,10);
\draw [short] (7,10) .. controls (7.5,9.75) and (7.5,9.5) .. (7.75,9.5);
\draw [short] (8.75,8) .. controls (9.25,9.25) and (9.25,8.25) .. (9.75,9);
\draw [short] (7.75,9.5) .. controls (7.5,9) and (7.75,8.75) .. (8,8.5);
\draw [short] (8,8.5) .. controls (8.75,8.75) and (8.5,8.25) .. (8.75,8);
\end{circuitikz}
}%
\end{enumerate}
\item A negative sequence relay is commonly used to protect 
\begin{enumerate}
    \item an alternator
    \item a transformer
    \item a transmission line
    \item a bus bar \\
\end{enumerate}
\item For enhancing the power transmission in along EHV transmission line, the most preferred method is to connect a 
\begin{enumerate}
    \item series inductive compensator in the line
    \item shunt inductive compensator at the receiving end 
    \item series capacitive compensator in the line
    \item shunt capacitive compensator at the sending end \\
\end{enumerate}
\item An open loop system represented by the transfer function $G\brak{s} = \frac{\brak{s-1}}{\brak{s+2}\brak{s+3}}$ is
 \begin{enumerate}
    \item stable and of the minimum phase type
    \item stable and of the non - minimum phase type 
    \item unstable and of the minimum phase type 
    \item unstable and of non-minimum phase type  \\
 \end{enumerate}
\item The bridge circuit shown in the figure below is used for the measurement of an unknown element $Z_X.$ The bridge circuit is best suited when $Z_X$ is a 
\begin{figure}[!ht]
\centering
\resizebox{0.5\textwidth}{!}{%
\begin{circuitikz}
\tikzstyle{every node}=[font=\normalsize]
\draw [short] (5,11.25) -- (5,7.5);
\draw [short] (8.75,11.25) -- (8.75,7.5);
\draw [short] (5,11.25) .. controls (6.75,13) and (7,9.75) .. (8.75,11.25);
\draw [short] (5,7.5) .. controls (7,9.25) and (7,6) .. (8.75,7.5);
\draw [<->, >=Stealth] (5,8.5) -- (8.75,8.5)node[pos=0.5, fill=white]{20 mm};
\node [font=\normalsize] at (4.25,10.25) {Left face};
\node [font=\normalsize] at (9.75,10.25) {Right face};
\node [font=\normalsize] at (4,9.25) {$T$ =150 \degree C };
\node [font=\normalsize] at (10,9.5) {$T$ = 110 \degree C};
\node [font=\normalsize] at (6.5,10) {$\dot{q}$ = 100 MW/$\text{m}^3$};
\end{circuitikz}

}%
\end{figure}
\begin{enumerate}
     \item low resistance 
     \item high resistance
     \item low Q inductor
     \item lossy capacitor \\
 \end{enumerate}
\item A dual trace oscilloscope is set to operate in the ALTernate mode. The control input of the multiplexer used in the y-circuit is fed with a signal having a frequency equal to 
\begin{enumerate}
    \item the highest frequency that the multiplexer can operate properly 
    \item twice the frequency of the time base (sweep) oscillator
    \item the frequency of the time base (sweep) oscillator
    \item haif the frequency of the time base (sweep) oscillator \\
\end{enumerate}
\item The output \textbf{Y} of the logic circuit given below is
\begin{figure}[!ht]
\centering
\resizebox{0.5\textwidth}{!}{%
\begin{circuitikz}
\tikzstyle{every node}=[font=\normalsize]
\draw (9,10) to[short] (9.25,10);
\draw (9,9.5) to[short] (9.25,9.5);
\draw (9.25,10) node[ieeestd xor port, anchor=in 1, scale=0.89](port){} (port.out) to[short] (11,9.75);
\draw (5,9.5) node[ieeestd not port, anchor=in](port){} (port.out) to[short] (7,9.5);
\draw (port.in) to[short] (4.5,9.5);
\draw [short] (7,9.5) -- (9,9.5);
\draw [short] (3.75,10) -- (9.25,10);
\draw [short] (4.5,10) -- (4.5,9.5);
\node [font=\normalsize] at (11.5,9.75) {\textbf{Y}};
\node [font=\normalsize] at (3.5,10) {\textbf{X}};
\end{circuitikz}

}%
\end{figure}
\begin{enumerate}
    \item 1
    \item 0
    \item \textbf{X}
    \item \textbf{$\overline{\text{X}}$} \\
\end{enumerate}
\item Circuit turn-off time of an SCR is defined as the time
  \begin{enumerate}
    \item taken by the SCR turn of 
    \item required for the SCR current to become zero 
    \item for which the SCR is reverse biased by the commutation circuit 
    \item for which the SCR is reverse biased to reduce its current below the holding current \\
\end{enumerate}
\item Solution of the variables $x_1$ and $x_2$ for the following equations is to be obtained by employing the Newton-Raphson iterative method.\\
equation \brak{\text{i}} \quad $10x_2\sin\brak{x_1}-0.8=0$ \\
equation \brak{\text{ii}} \quad $10x_2^2-10x_2\cos\brak{x_1}-0.6=0$\\
Assuming the initial valued $x_1$ = 0.0 and $x_2$ = 1.0, the jacobian matrix is 
\begin{enumerate}
    \item \sbrak{
    \begin{matrix}
        10 & -0.8 \\ 0 & -0.6
    \end{matrix}
    }
     \item \sbrak{
    \begin{matrix}
        10 & 0 \\ 0 & 10
    \end{matrix}
    }
     \item \sbrak{
    \begin{matrix}
        0 & -0.8 \\ 10 & -0.6
    \end{matrix}
    }
     \item \sbrak{
    \begin{matrix}
        10 & 0 \\ 10 & -10
    \end{matrix}
    } \\
\end{enumerate}
			 \end{enumerate}
			 \end{document}
 g
