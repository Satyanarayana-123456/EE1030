\let\negmedspace\undefined
\let\negthickspace\undefined
\documentclass[journal]{IEEEtran}
\usepackage[a5paper, margin=10mm, onecolumn]{geometry}
%\usepackage{lmodern} % Ensure lmodern is loaded for pdflatex
\usepackage{tfrupee} % Include tfrupee package

\setlength{\headheight}{1cm} % Set the height of the header box
\setlength{\headsep}{0mm}     % Set the distance between the header box and the top of the text

\usepackage{gvv-book}
\usepackage{gvv}
\usepackage{cite}
\usepackage{amsmath,amssymb,amsfonts,amsthm}
\usepackage{algorithmic}
\usepackage{graphicx}
\usepackage{textcomp}
\usepackage{xcolor}
\usepackage{txfonts}
\usepackage{listings}
\usepackage{enumitem}
\usepackage{mathtools}
\usepackage{gensymb}
\usepackage{comment}
\usepackage[breaklinks=true]{hyperref}
\usepackage{tkz-euclide} 
\usepackage{listings}
% \usepackage{gvv}                                        
\def\inputGnumericTable{}                                 
\usepackage[latin1]{inputenc}                                
\usepackage{color}                                            
\usepackage{array}                                            
\usepackage{longtable}                                       
\usepackage{calc}                                             
\usepackage{multirow}                                         
\usepackage{hhline}                                           
\usepackage{ifthen}                                           
\usepackage{lscape}
\begin{document}

\bibliographystyle{IEEEtran}
\vspace{3cm}




\title{
%	\logo{
GATE - 2023 - XE

\large{EE1030 : Matrix Theory}

Indian Institute of Technology Hyderabad
%	}
}
\author{Satyanarayana Gajjarapu

AI24BTECH11009
}	





\maketitle




\bigskip

\renewcommand{\thefigure}{\theenumi}
\renewcommand{\thetable}{\theenumi}


\section{40 - 52}


\begin{enumerate}
\item Water (density = 1000 kg/$\text{m}^3$) flows steadily with a flow rate of 0.05 $\text{m}^3$/s through a venturimeter having throat diameter of 100 mm. If the pipe diameter is 200 mm and losses are negligible, the pressure drop (in kPa, $rounded\ off\ to\ one\ decimal\ place$) between an upstream location in the pipe and the throat (both at the same elevation) is $\_\_\_\_$. \\
\item Water flows around a thin flat plate (0.25 m long, 2 m wide) with a free stream velocity $\brak{U_{\infty}}$ of 1 m/s, as shown in the figure. Consider linear velocity profile $\brak{\frac{u}{U_{\infty}} = \frac{y}{\delta}}$ for which the laminar boundary layer thickness is expressed as $\delta = \frac{3.5x}{\sqrt{Re_x}}$. For water, density = 1000 kg/$\text{m}^3$ and dynamic viscosity = 0.001 kg/m.s. Net drag force (in N, $rounded\ off\ to\ two\ decimal\ places$) acting on the plate, neglecting the end effects, is $\_\_\_\_$. 
\begin{figure}[!ht]
\centering
\resizebox{0.7\textwidth}{!}{%
\begin{circuitikz}
\tikzstyle{every node}=[font=\normalsize]
\draw [->, >=Stealth] (6.25,9.5) -- (6.25,11);
\draw [->, >=Stealth] (6.25,9.5) -- (14.5,9.5);
\draw [short] (6.25,9.25) -- (6.25,8.75);
\draw [short] (12.75,9.25) -- (12.75,8.75);
\draw [<->, >=Stealth] (6.25,9) -- (12.75,9);
\draw [line width=1.6pt, short] (6.25,9.5) -- (12.75,9.5);
\draw [->, >=Stealth] (2.5,10.75) -- (5,10.75);
\draw [->, >=Stealth] (2.5,10) -- (5,10);
\draw [->, >=Stealth] (2.5,9.25) -- (5,9.25);
\draw [->, >=Stealth] (2.5,8.5) -- (5,8.5);
\node [font=\normalsize] at (9.25,8.75) {0.25 m};
\node [font=\normalsize] at (14.75,9.5) {$x$};
\node [font=\normalsize] at (6.25,11.25) {$y$};
\node [font=\normalsize] at (3.5,11) {$U_{\infty}$};
\end{circuitikz}

}%
\end{figure} \\
\item Axial velocity profile $u\brak{r}$ for an axisymmetric flow through a circular tube of radius $R$ is given as,
\begin{align*}
    \frac{u\brak{r}}{U} = \brak{1 - \frac{r}{R}}^{\frac{1}{n}}
\end{align*}
where $U$ is the centerline velocity. If $V$ refers to the area-averaged velocity (volume flow rate per unit area), then the ratio $\frac{V}{U}$ for $n = 1$ \brak{rounded\ off\ to\ two\ decimal\ places} is $\_\_\_\_$. \\
\item A stationary circular pipe of radius $R = 0.5$ m is half filled withwater (density = 1000 kg/$\text{m}^3$), whereas the upper half is filled with air at atmospheric pressure, as shown in the figure. Acceleration due to gravity is $g$ = 9.81 m/$\text{s}^2$. The magnitude of the force per unit length (in kN/m, $rounded\ off\ to\ one\ decimal\ place$) applied by water on the pipe section AB is $\_\_\_\_$.
\begin{figure}[!ht]
\centering
\resizebox{0.5\textwidth}{!}{%
\begin{circuitikz}
\tikzstyle{every node}=[font=\normalsize]
\draw [short] (5,10.5) -- (5,9.25);
\draw [short] (5,9.25) -- (11.5,9.25);
\draw [short] (11.5,10.5) -- (11.5,9.25);
\draw [short] (5,7.5) -- (11.5,7.5);
\draw [short] (11.5,7.5) -- (11.5,6.25);
\draw [short] (5,7.5) -- (5,6.25);
\draw [->, >=Stealth] (5,9) -- (6.25,9);
\draw [->, >=Stealth] (5,8.75) -- (6.25,8.75);
\draw [->, >=Stealth] (5,8.25) -- (6.25,8.25);
\draw [->, >=Stealth] (5,7.75) -- (6.25,7.75);
\draw [->, >=Stealth] (5,8) -- (6.25,8);
\draw [->, >=Stealth] (5,8.5) -- (6.25,8.5);
\draw [short] (11.5,9.25) .. controls (14,8.5) and (12.75,8) .. (11.5,7.5);
\draw [->, >=Stealth] (1.5,8) -- (1.5,9.75);
\draw [->, >=Stealth] (1.5,8) -- (3,8);
\draw [->, >=Stealth] (11.5,9) -- (12,9);
\draw [->, >=Stealth] (11.5,8.75) -- (12.5,8.75);
\draw [->, >=Stealth] (11.5,8.5) -- (12.75,8.5);
\draw [->, >=Stealth] (11.5,8.25) -- (12.75,8.25);
\draw [->, >=Stealth] (11.5,8) -- (12.5,8);
\draw [->, >=Stealth] (11.5,7.75) -- (12.25,7.75);
\draw [line width=0.2pt, short] (11.5,9.25) -- (11.5,7.5);
\draw [short] (5,9.25) -- (6.5,10.5);
\draw [short] (6,9.25) -- (7.5,10.25);
\draw [short] (7.25,9.25) -- (8.5,10.25);
\draw [short] (8.5,9.25) -- (9.75,10.25);
\draw [short] (9.75,9.25) -- (10.75,10.25);
\draw [short] (10.75,9.25) -- (11.5,10);
\draw [short] (5.5,9.25) -- (7,10.5);
\draw [short] (6.75,9.25) -- (8,10.25);
\draw [short] (8,9.25) -- (9.25,10.5);
\draw [short] (9.25,9.25) -- (10.25,10.25);
\draw [short] (10.25,9.25) -- (11.25,10.25);
\draw [short] (5,7) -- (5.75,7.5);
\draw [short] (5,6.5) -- (6.75,7.5);
\draw [short] (6,6.75) -- (7.5,7.5);
\draw [short] (7.25,6.5) -- (8.75,7.5);
\draw [short] (8.25,6.5) -- (9.5,7.5);
\draw [short] (9,6.25) -- (10,7.5);
\draw [short] (10.75,6.5) -- (11.5,7.5);
\draw [short] (10,6.5) -- (10.75,7.5);
\draw [short] (6.5,6.5) -- (8,7.5);
\node [font=\normalsize] at (12.25,9.5) {$y = y_0$};
\node [font=\normalsize] at (12.25,7.25) {$y = 0$};
\node [font=\normalsize] at (4.5,8.5) {$U_0$};
\node [font=\normalsize] at (3,7.75) {$x$};
\node [font=\normalsize] at (1.75,10) {$y$};
\end{circuitikz}

}%
\end{figure} \\
\item In age-hardening of an aluminium alloy, the purpose of solution treatment
followed by quenching is to
\begin{enumerate}
    \item form martensitic structure
    \item increase the size of the precipitates
    \item form supersaturated solid solution
    \item form precipitates at the grain boundaries \\
\end{enumerate}
\item The magnetization (M) - magnetic field (H) curves for four different materials are given below. Which one of these materials is most suitable for use as a permanent magnet?
\begin{enumerate}
    \item \resizebox{0.4\textwidth}{!}{%
\begin{circuitikz}
\tikzstyle{every node}=[font=\normalsize]
\draw [short] (7.25,12.5) -- (7.25,8);
\draw [short] (4.25,10.25) -- (10.25,10.25);
\draw [short] (7,10.75) -- (7.5,10.75);
\draw [short] (7,11.25) -- (7.5,11.25);
\draw [short] (7,11.75) -- (7.5,11.75);
\draw [short] (7,12.25) -- (7.5,12.25);
\draw [short] (8,10.5) -- (8,10);
\draw [short] (8.75,10.5) -- (8.75,10);
\draw [short] (9.5,10.5) -- (9.5,10);
\draw [short] (10.25,10.5) -- (10.25,10);
\node [font=\normalsize] at (7.5,10) {0};
\node [font=\normalsize] at (8,9.75) {1};
\node [font=\normalsize] at (6.75,10.75) {1};
\node [font=\normalsize] at (9.5,9.75) {3};
\node [font=\normalsize] at (6.75,11.75) {3};
\node [font=\normalsize] at (7.25,12.75) {M};
\node [font=\normalsize] at (10.5,10) {H};
\draw [ dashed] (6.5,11.5) rectangle  (8.25,9);
\end{circuitikz}

}%
\item \resizebox{0.4\textwidth}{!}{%
\begin{circuitikz}
\tikzstyle{every node}=[font=\normalsize]
\draw [short] (7.25,12.5) -- (7.25,8);
\draw [short] (4.25,10.25) -- (10.25,10.25);
\draw [short] (7,10.75) -- (7.5,10.75);
\draw [short] (7,11.25) -- (7.5,11.25);
\draw [short] (7,11.75) -- (7.5,11.75);
\draw [short] (7,12.25) -- (7.5,12.25);
\draw [short] (8,10.5) -- (8,10);
\draw [short] (8.75,10.5) -- (8.75,10);
\draw [short] (9.5,10.5) -- (9.5,10);
\draw [short] (10.25,10.5) -- (10.25,10);
\node [font=\normalsize] at (7.5,10) {0};
\node [font=\normalsize] at (8,9.75) {1};
\node [font=\normalsize] at (6.75,10.75) {1};
\node [font=\normalsize] at (9.5,9.75) {3};
\node [font=\normalsize] at (6.75,11.75) {3};
\node [font=\normalsize] at (7.25,12.75) {M};
\node [font=\normalsize] at (10.5,10) {H};
\draw [ dashed] (6.5,11.5) rectangle  (7.75,9.25);
\end{circuitikz}

}%
\item \resizebox{0.4\textwidth}{!}{%
\begin{circuitikz}
\tikzstyle{every node}=[font=\normalsize]
\draw [short] (7.25,12.5) -- (7.25,8);
\draw [short] (4.25,10.25) -- (10.25,10.25);
\draw [short] (7,10.75) -- (7.5,10.75);
\draw [short] (7,11.25) -- (7.5,11.25);
\draw [short] (7,11.75) -- (7.5,11.75);
\draw [short] (7,12.25) -- (7.5,12.25);
\draw [short] (8,10.5) -- (8,10);
\draw [short] (8.75,10.5) -- (8.75,10);
\draw [short] (9.5,10.5) -- (9.5,10);
\draw [short] (10.25,10.5) -- (10.25,10);
\node [font=\normalsize] at (7.5,10) {0};
\node [font=\normalsize] at (8,9.75) {1};
\node [font=\normalsize] at (6.75,10.75) {1};
\node [font=\normalsize] at (9.5,9.75) {3};
\node [font=\normalsize] at (6.75,11.75) {3};
\node [font=\normalsize] at (7.25,12.75) {M};
\node [font=\normalsize] at (10.5,10) {H};
\draw [ dashed] (5.75,12) rectangle  (9.25,8.5);
\end{circuitikz}

}%
\item \resizebox{0.4\textwidth}{!}{%
\begin{circuitikz}
\tikzstyle{every node}=[font=\normalsize]
\draw [short] (7.25,12.5) -- (7.25,8);
\draw [short] (4.25,10.25) -- (10.25,10.25);
\draw [short] (7,10.75) -- (7.5,10.75);
\draw [short] (7,11.25) -- (7.5,11.25);
\draw [short] (7,11.75) -- (7.5,11.75);
\draw [short] (7,12.25) -- (7.5,12.25);
\draw [short] (8,10.5) -- (8,10);
\draw [short] (8.75,10.5) -- (8.75,10);
\draw [short] (9.5,10.5) -- (9.5,10);
\draw [short] (10.25,10.5) -- (10.25,10);
\node [font=\normalsize] at (7.5,10) {0};
\node [font=\normalsize] at (8,9.75) {1};
\node [font=\normalsize] at (6.75,10.75) {1};
\node [font=\normalsize] at (9.5,9.75) {3};
\node [font=\normalsize] at (6.75,11.75) {3};
\node [font=\normalsize] at (7.25,12.75) {M};
\node [font=\normalsize] at (10.5,10) {H};
\draw [ dashed] (6.5,11) rectangle  (7.75,9.5);
\end{circuitikz}


}%
\\
\end{enumerate}
\item The band gap of a semiconducting material is $\sim$ 2 eV. Which one of the following absorption (A) vs. energy (in eV) curves is correct ?
\begin{enumerate}
    \item \resizebox{0.4\textwidth}{!}{%
\begin{circuitikz}
\tikzstyle{every node}=[font=\normalsize]
\draw [->, >=Stealth] (7.25,9) -- (7.25,11.75);
\draw [->, >=Stealth] (7.25,9) -- (10.5,9);
\draw [short] (8,9.25) -- (8,8.75);
\draw [short] (8.75,9.25) -- (8.75,8.75);
\draw [short] (9.5,9.25) -- (9.5,8.75);
\node [font=\normalsize] at (7,10.25) {A};
\node [font=\normalsize] at (8.75,8) {Energy(eV)};
\node [font=\normalsize] at (8,8.5) {1};
\node [font=\normalsize] at (8.75,8.5) {2};
\node [font=\normalsize] at (9.5,8.5) {3};
\node [font=\normalsize] at (6.75,11) {90 \%};
\draw [short] (8.5,10.5) .. controls (8.5,9.75) and (8.5,9.25) .. (9,9);
\draw [short] (7.25,11) .. controls (8.25,11.25) and (8.5,11.25) .. (8.5,10.5);
\end{circuitikz}

}%
\item \resizebox{0.4\textwidth}{!}{%
\begin{circuitikz}
\tikzstyle{every node}=[font=\normalsize]
\draw [->, >=Stealth] (7.25,9) -- (7.25,11.75);
\draw [->, >=Stealth] (7.25,9) -- (10.5,9);
\draw [short] (8,9.25) -- (8,8.75);
\draw [short] (8.75,9.25) -- (8.75,8.75);
\draw [short] (9.5,9.25) -- (9.5,8.75);
\node [font=\normalsize] at (7,10.25) {A};
\node [font=\normalsize] at (8.75,8) {Energy(eV)};
\node [font=\normalsize] at (8,8.5) {1};
\node [font=\normalsize] at (8.75,8.5) {2};
\node [font=\normalsize] at (9.5,8.5) {3};
\node [font=\normalsize] at (6.75,11) {90 \%};
\draw [short] (7.25,11) -- (9.75,11);
\end{circuitikz}

}%
\item \resizebox{0.4\textwidth}{!}{%
\begin{circuitikz}
\tikzstyle{every node}=[font=\normalsize]
\draw [->, >=Stealth] (7.25,9) -- (7.25,11.75);
\draw [->, >=Stealth] (7.25,9) -- (10.5,9);
\draw [short] (8,9.25) -- (8,8.75);
\draw [short] (8.75,9.25) -- (8.75,8.75);
\draw [short] (9.5,9.25) -- (9.5,8.75);
\node [font=\normalsize] at (7,10.25) {A};
\node [font=\normalsize] at (8.75,8) {Energy(eV)};
\node [font=\normalsize] at (8,8.5) {1};
\node [font=\normalsize] at (8.75,8.5) {2};
\node [font=\normalsize] at (9.5,8.5) {3};
\draw [short] (7.25,9.5) -- (10,9.5);
\node [font=\normalsize] at (6.75,9.5) {10 \%};
\end{circuitikz}

}%
\item \resizebox{0.4\textwidth}{!}{%
\begin{circuitikz}
\tikzstyle{every node}=[font=\normalsize]
\draw [->, >=Stealth] (7.25,9) -- (7.25,11.75);
\draw [->, >=Stealth] (7.25,9) -- (10.5,9);
\draw [short] (8,9.25) -- (8,8.75);
\draw [short] (8.75,9.25) -- (8.75,8.75);
\draw [short] (9.5,9.25) -- (9.5,8.75);
\node [font=\normalsize] at (7,10.25) {A};
\node [font=\normalsize] at (8.75,8) {Energy(eV)};
\node [font=\normalsize] at (8,8.5) {1};
\node [font=\normalsize] at (8.75,8.5) {2};
\node [font=\normalsize] at (9.5,8.5) {3};
\node [font=\normalsize] at (6.75,11) {90 \%};
\draw [short] (8.5,9) .. controls (9.25,9.5) and (9,9.75) .. (9,10.75);
\draw [short] (9,10.75) .. controls (9,11.5) and (9.5,11) .. (10,11);
\end{circuitikz}

}%
\\
\end{enumerate}
\item Figures (i) and (ii) show a binary phase diagram and the corresponding Gibbs free energy \brak{G} vs. composition \brak{X_B} diagram, respectively. Figure (ii) corresponds to which one of the temperatures shown in Figure (i)?
\pagebreak
\begin{figure}[!ht]
\centering
\resizebox{0.9\textwidth}{!}{%
\begin{tabular}{|c|c|c|c|c|}
\hline
$x$ & $x_1 = 2$ & $x_2 = 6$ & $x_3 = 8$ & $x_4 = 9$ \\
\hline
$f$ & $4$ & $4$ & $\alpha$ & $\beta$ \\
\hline
\end{tabular}


}%
\end{figure}
\begin{enumerate}
    \item T1
    \item T2
    \item T3
    \item T4 \\
\end{enumerate}
\item Aliovalent doping of $MgCl_2$ in $NaCl$ leads to the formation of defects. Which one of the following is the correct defect reaction ?
\begin{enumerate}
    \item $Mg^{\bullet}_{Cl} + Na_{Na} + V'_{Cl} = \varnothing$
    \item $Mg^{\bullet}_{Na} + Cl_{Cl} + V'_{Na} = \varnothing$
    \item $Mg_{Na} + Cl_{Cl} = \varnothing$
    \item $Mg'_{Na} + Cl_{Cl} + V^{\bullet}_{Na} = \varnothing$ \\
\end{enumerate}
\item A screw dislocation in a FCC crystal has Burgers vector of $\frac{a}{2}\sbrak{1 1 0}$, where $a$ is the lattice constant. The possible slip plane(s) is/are:
\begin{enumerate}
    \item $\brak{1 1 \bar{1}}$
    \item $\brak{1 1 1}$
    \item $\brak{\bar{1} 1 1}$
    \item $\brak{1 \bar{1} 1}$ \\
\end{enumerate}
\item The tensile true stress $\brak{\sigma}$ - true strain $\brak{\epsilon}$ curve follows the Hollomon equation:
\begin{align*}
    \sigma = 500\epsilon^{0.15}\ \text{MPa}
\end{align*}
At the maximum load, the work-hardening rate $\brak{\frac{d\sigma}{d\epsilon}}$ is (in MPa): $\_\_\_\_$ (rounded off to nearest integer) \\
\item A metal has a certain vacancy fraction at a temperature of 600 K. On increasing the temperature to 900 K, the vacancy fraction increases by a factor of $\_\_\_\_$ (rounded off to one decimal place) \\\\
Given: Gas constant, R = 8.31 J $\text{mol}^{-1}\text{K}^{-1}$ and activation energy for vacancy formation, Q = 68 kJ $\text{mol}^{-1}$ \\
\item In a semiconductor, the ratio of electronic mobility to hole mobility is 10. The density of electrons and holes are $10^{15} m^{-3}$ and $10^{16} m^{-3}$, respectively. If the conductivity of the material is 1.6 $\Omega^{-1} m^{-1}$, then the mobility of holes is (in $m^2V^{-1}s^{-1}$) : $\_\_\_\_$ (rounded off to nearest integer) \\\\
Given: Charge of an electron: $1.6 \times 10^{-19}\ C$ \\
			 \end{enumerate}
			 \end{document}
  
