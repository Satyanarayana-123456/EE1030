\let\negmedspace\undefined
\let\negthickspace\undefined
\documentclass[journal]{IEEEtran}
\usepackage[a5paper, margin=10mm, onecolumn]{geometry}
%\usepackage{lmodern} % Ensure lmodern is loaded for pdflatex
\usepackage{tfrupee} % Include tfrupee package

\setlength{\headheight}{1cm} % Set the height of the header box
\setlength{\headsep}{0mm}     % Set the distance between the header box and the top of the text

\usepackage{gvv-book}
\usepackage{gvv}
\usepackage{cite}
\usepackage{amsmath,amssymb,amsfonts,amsthm}
\usepackage{algorithmic}
\usepackage{graphicx}
\usepackage{textcomp}
\usepackage{xcolor}
\usepackage{txfonts}
\usepackage{listings}
\usepackage{enumitem}
\usepackage{mathtools}
\usepackage{gensymb}
\usepackage{comment}
\usepackage[breaklinks=true]{hyperref}
\usepackage{tkz-euclide} 
\usepackage{listings}
% \usepackage{gvv}                                        
\def\inputGnumericTable{}                                 
\usepackage[latin1]{inputenc}                                
\usepackage{color}                                            
\usepackage{array}                                            
\usepackage{longtable}                                       
\usepackage{calc}                                             
\usepackage{multirow}                                         
\usepackage{hhline}                                           
\usepackage{ifthen}                                           
\usepackage{lscape}
\begin{document}

\bibliographystyle{IEEEtran}
\vspace{3cm}




\title{
%	\logo{
GATE - 2013 - ME

\large{EE1030 : Matrix Theory}

Indian Institute of Technology Hyderabad
%	}
}
\author{Satyanarayana Gajjarapu

AI24BTECH11009
}	





\maketitle




\bigskip

\renewcommand{\thefigure}{\theenumi}
\renewcommand{\thetable}{\theenumi}


\section{1 - 13}


\begin{enumerate}
\item The partial differential equation $\frac{\partial u}{\partial t} + u\frac{\partial u}{\partial x} = \frac{\partial^2 u}{\partial x^2}$ is a
    \begin{enumerate}
      \item linear equation of order 2 
      \item non-linear equation of order 1
      \item linear equation of order 1
      \item non-linear equation of order 2  \\
    \end{enumerate}
\item The eigenvalues of a symmetric matrix are all  
\begin{enumerate}
    \item complex with non-zero positive imaginary part. 
    \item complex with non-zero negative imaginary part.
    \item real. 
    \item pure imaginary. \\
\end{enumerate}
\item Match the \textbf{CORRECT} pairs.
\begin{table}[h!]
  \centering
  \begin{circuitikz}
\tikzstyle{every node}=[font=\normalsize]
\draw (5.75,10.75) to[R] (8.5,10.75);
\draw (5.75,11.75) to[R] (8.5,11.75);
\draw (5.75,11.75) to[short] (5.75,10.75);
\draw (8.5,11.75) to[short] (8.5,10.75);
\draw (8.5,11.25) to[short] (9.75,11.25);
\draw (5.75,11.25) to[short] (4.75,11.25);
\draw (9.75,11.25) to[R] (9.75,8.75);
\draw (4.75,11.25) to[battery1] (4.75,8.75);
\draw [->, >=Stealth] (6.5,11.5) -- (7.75,12);
\draw [short] (9.75,8.75) -- (4.75,8.75);
\node [font=\normalsize] at (5,10.25) {$+$};
\node [font=\normalsize] at (4,10) {10 V};
\node [font=\normalsize] at (7,12.5) {R};
\node [font=\normalsize] at (7,10.25) {6 $\Omega$};
\node [font=\normalsize] at (8.75,10) {3 $\Omega$};
\draw [ dashed] (9.25,10.75) rectangle  (10.25,9.25);
\node at (5.75,11.25) [circ] {};
\node at (8.5,11.25) [circ] {};
\node [font=\normalsize] at (10.75,10) {Load};
\end{circuitikz}

\end{table}
\begin{enumerate}
    \item P-2, Q-1, R-3 
    \item P-3, Q-2, R-1
    \item P-1, Q-2, R-3 
    \item P-3, Q-1, R-2  \\
\end{enumerate}
\item A rod of length $L$ having uniform cross-sectional area $A$ is subjected to a tensile force $P$ as shown in the figure below. If the Young's modulus of the material varies linearly from $E_1$ to $E_2$ along the length of the rod, the normal stress developed at the section-SS is 
\pagebreak
\begin{figure}[!ht]
\centering
\resizebox{0.5\textwidth}{!}{%
\begin{circuitikz}
\tikzstyle{every node}=[font=\normalsize]
\draw [short] (5,10.5) -- (5,9.25);
\draw [short] (5,9.25) -- (11.5,9.25);
\draw [short] (11.5,10.5) -- (11.5,9.25);
\draw [short] (5,7.5) -- (11.5,7.5);
\draw [short] (11.5,7.5) -- (11.5,6.25);
\draw [short] (5,7.5) -- (5,6.25);
\draw [->, >=Stealth] (5,9) -- (6.25,9);
\draw [->, >=Stealth] (5,8.75) -- (6.25,8.75);
\draw [->, >=Stealth] (5,8.25) -- (6.25,8.25);
\draw [->, >=Stealth] (5,7.75) -- (6.25,7.75);
\draw [->, >=Stealth] (5,8) -- (6.25,8);
\draw [->, >=Stealth] (5,8.5) -- (6.25,8.5);
\draw [short] (11.5,9.25) .. controls (14,8.5) and (12.75,8) .. (11.5,7.5);
\draw [->, >=Stealth] (1.5,8) -- (1.5,9.75);
\draw [->, >=Stealth] (1.5,8) -- (3,8);
\draw [->, >=Stealth] (11.5,9) -- (12,9);
\draw [->, >=Stealth] (11.5,8.75) -- (12.5,8.75);
\draw [->, >=Stealth] (11.5,8.5) -- (12.75,8.5);
\draw [->, >=Stealth] (11.5,8.25) -- (12.75,8.25);
\draw [->, >=Stealth] (11.5,8) -- (12.5,8);
\draw [->, >=Stealth] (11.5,7.75) -- (12.25,7.75);
\draw [line width=0.2pt, short] (11.5,9.25) -- (11.5,7.5);
\draw [short] (5,9.25) -- (6.5,10.5);
\draw [short] (6,9.25) -- (7.5,10.25);
\draw [short] (7.25,9.25) -- (8.5,10.25);
\draw [short] (8.5,9.25) -- (9.75,10.25);
\draw [short] (9.75,9.25) -- (10.75,10.25);
\draw [short] (10.75,9.25) -- (11.5,10);
\draw [short] (5.5,9.25) -- (7,10.5);
\draw [short] (6.75,9.25) -- (8,10.25);
\draw [short] (8,9.25) -- (9.25,10.5);
\draw [short] (9.25,9.25) -- (10.25,10.25);
\draw [short] (10.25,9.25) -- (11.25,10.25);
\draw [short] (5,7) -- (5.75,7.5);
\draw [short] (5,6.5) -- (6.75,7.5);
\draw [short] (6,6.75) -- (7.5,7.5);
\draw [short] (7.25,6.5) -- (8.75,7.5);
\draw [short] (8.25,6.5) -- (9.5,7.5);
\draw [short] (9,6.25) -- (10,7.5);
\draw [short] (10.75,6.5) -- (11.5,7.5);
\draw [short] (10,6.5) -- (10.75,7.5);
\draw [short] (6.5,6.5) -- (8,7.5);
\node [font=\normalsize] at (12.25,9.5) {$y = y_0$};
\node [font=\normalsize] at (12.25,7.25) {$y = 0$};
\node [font=\normalsize] at (4.5,8.5) {$U_0$};
\node [font=\normalsize] at (3,7.75) {$x$};
\node [font=\normalsize] at (1.75,10) {$y$};
\end{circuitikz}

}%
\end{figure}
 \begin{enumerate}
     \item $\frac{P}{A}$
     \item $\frac{P\brak{E_1-E_2}}{A\brak{E_1+_2}}$
     \item $\frac{PE_2}{AE_1}$
     \item $\frac{PE_1}{AE_2}$ \\
 \end{enumerate}
\item Two threaded bolts A and B of same material and length are subjected to identical tensile load. If the elastic strain energy stored in bolt A is 4 times that of bolt B and the mean diameter of bolt A is 12 $mm$, the mean diameter of bolt B in $mm$ is 
\begin{enumerate}
   \item 16
   \item 24
   \item 36
   \item 48 \\
\end{enumerate}
\item A link OB is rotating with a constant angular velocity of 2 $rad/s$ in counter clockwise direction and a block is sliding radially outward on it with an uniform velocity of 0.75 $m/s$ with respect to the rod, as shown in the figure below. If OA = 1 $m$, the magnitude of the absolute acceleration of the
block at location A in $m/s^2$ is 
\begin{figure}[!ht]
\centering
\resizebox{0.25\textwidth}{!}{%
\begin{circuitikz}
\tikzstyle{every node}=[font=\normalsize]
\draw [short] (3.5,11.25) -- (3.5,9);
\draw [short] (3.25,11.25) -- (3.25,9);
\draw [short] (3.25,11.25) -- (3.5,11.25);
\draw [short] (3.25,9) -- (3.5,9);
\draw [short] (3.25,11.25) -- (3,11);
\draw [short] (3.25,11) -- (3,10.75);
\draw [short] (3.25,10.75) -- (3,10.5);
\draw [short] (3.25,10.5) -- (3,10.25);
\draw [short] (3.25,10.25) -- (3,10);
\draw [short] (3.25,10) -- (3,9.75);
\draw [short] (3.25,9.75) -- (3,9.5);
\draw [short] (3.25,9.5) -- (3,9.25);
\draw [short] (3.25,9.25) -- (3,9);
\draw  (3.5,10.25) rectangle (11,10);
\draw  (11,11.25) rectangle (11.25,9);
\draw [short] (11.25,11.25) -- (11.5,11);
\draw [short] (11.25,11) -- (11.5,10.75);
\draw [short] (11.25,10.5) -- (11.5,10.25);
\draw [short] (11.25,10.75) -- (11.5,10.5);
\draw [short] (11.25,10) -- (11.5,9.75);
\draw [short] (11.25,10.25) -- (11.5,10);
\draw [short] (11.25,9.75) -- (11.5,9.5);
\draw [short] (11.25,9.5) -- (11.5,9.25);
\draw [short] (11.25,9.25) -- (11.5,9);
\draw [->, >=Stealth] (7.25,11.75) -- (7.25,10.25);
\draw [short] (5.75,11) -- (5.75,10.5);
\draw [short] (8.75,11) -- (8.75,10.5);
\draw [<->, >=Stealth] (3.5,10.75) -- (5.75,10.75);
\draw [<->, >=Stealth] (5.75,10.75) -- (7.25,10.75);
\draw [<->, >=Stealth] (7.25,10.75) -- (8.75,10.75);
\draw [<->, >=Stealth] (8.75,10.75) -- (11,10.75);
\draw [short] (5.75,10) -- (5.5,9.75);
\draw [short] (5.75,10) -- (6,9.75);
\draw [short] (5.25,9.75) -- (6.25,9.75);
\draw [short] (8.75,10) -- (8.5,9.75);
\draw [short] (8.75,10) -- (9,9.75);
\draw [short] (8.25,9.75) -- (9.25,9.75);
\draw [short] (5.25,9.75) -- (5.5,9.5);
\draw [short] (5.5,9.75) -- (5.75,9.5);
\draw [short] (5.75,9.75) -- (6,9.5);
\draw [short] (6,9.75) -- (6.25,9.5);
\draw [short] (8.5,9.75) -- (8.75,9.5);
\draw [short] (8.75,9.75) -- (9,9.5);
\draw [short] (9,9.75) -- (9.25,9.5);
\draw [short] (8.25,9.75) -- (8.5,9.5);
\draw [->, >=Stealth] (8,8.25) -- (7.25,10);
\draw [->, >=Stealth] (2.5,9.5) .. controls (1.75,10.25) and (2,10.25) .. (2.5,11) ;
\node [font=\large] at (2.5,11.25) {$M$};
\node [font=\normalsize] at (4.75,11) {3 m};
\node [font=\normalsize] at (6.5,11) {1 m};
\node [font=\normalsize] at (8,11) {1 m};
\node [font=\normalsize] at (9.75,11) {3 m};
\node [font=\normalsize] at (7.25,12) {20 kN};
\node [font=\normalsize] at (4.25,9.75) {$EI$};
\node [font=\normalsize] at (6.75,9.75) {$EI$};
\node [font=\normalsize] at (10,9.75) {$EI$};
\node [font=\normalsize] at (8,8) {Internal hinge};
\draw [short] (7,10) -- (7.5,10.25);
\draw [short] (7,10.25) -- (7.5,10);
\end{circuitikz}

}%
\end{figure}
\begin{enumerate}
    \item 3
    \item 4
    \item 5
    \item 6 \\
\end{enumerate}
\item For steady, fully developed flow inside a straight pipe of diameter $D$, neglecting gravity effects, the pressure drop $\Delta p$ over a length $L$ and the wall shear stress $\tau_w$ are related by 
\begin{enumerate}
    \item $\tau_w = \frac{\Delta pD}{4L}$
    \item $\tau_w = \frac{\Delta pD^2}{4L^2}$
    \item $\tau_w = \frac{\Delta pD}{2L}$
    \item $\tau_w = \frac{4\Delta pL}{D}$ \\
 \end{enumerate}
\item The pressure, dry bulb temperature and relative humidity of air in a room are 1 $bar$, 30\degree C and 70\%, respectively. If the saturated steam pressure at 30\degree C is 4.25 $kPa$, the specific humidity of the room air in $kg\ water\ vapour/kg\ dry\ air$ is 
 \begin{enumerate}
    \item 0.0083
    \item 0.0101 
    \item 0.0191 
    \item 0.0232 \\
 \end{enumerate}
\item Consider one-dimensional steady state heat conduction, without heat generation, in a plane wall; with boundary conditions as shown in the figure below. The conductivity of the wall is given by $k = k_0 + bT$; where $k_0$ and $b$ are positive constants, and $T$ is temperature. 
\begin{figure}[!ht]
\centering
\resizebox{0.4\textwidth}{!}{%
\begin{tabular}[12pt]{ |c| c|}
    \hline
    \textbf{Type of Stress} & \textbf{Location}\\ 
    \hline
    P. Load & 1. Corner \\
    \hline 
    Q. Temperature & 2. Edge \\
    \hline
     & 3. Interior \\
    \hline
    \end{tabular}

}%
\end{figure}
As $x$ increases, the temperature gradient $\brak{\frac{dT}{dx}}$ will
\begin{enumerate}
     \item remain constant
     \item be zero
     \item increase
     \item decrease \\
 \end{enumerate}
\item In a rolling process, the state of stress of the material undergoing deformation is 
\begin{enumerate}
    \item pure compression
    \item pure shear
    \item compression and shear 
    \item tension and shear  \\
\end{enumerate}
\item Match the \textbf{CORRECT} pairs.
\begin{table}[h!]
  \centering
  \begin{tabular}[12pt]{ |c| c|}
    \hline
    \textbf{Processes} & \textbf{Characteristics / Applications} \\ 
    \hline
    P. Friction Welding & 1. Non-consumable electrode  \\
    \hline 
    Q. Gas Metal Arc Welding & 2. Joining of thick plates \\
    \hline
    R. Tungsten Inert Gas Welding & 3. Consumable electrode wire \\
    \hline
    S. Electroslag Welding & 4. Joining of cylindrical dissimilar materials \\
    \hline
    \end{tabular}

\end{table}
\begin{enumerate}
    \item P-4, Q-3, R-1, S-2
    \item P-4, Q-2, R-3, S-1
    \item P-2, Q-3, R-4, S-1 
    \item P-2, Q-4, R-1, S-3 \\
\end{enumerate}
\item A metric thread of pitch 2 $mm$ and thread angle 60\degree is inspected for its pitch diameter using 3-wire method. The diameter of the best size wire in $mm$ is 
  \begin{enumerate}
   \item 0.866
   \item 1.000
   \item 1.154
   \item 2.000 \\
\end{enumerate}
\item Customers arrive at a ticket counter at a rate of 50 per $hr$ and tickets are issued in the order of their arrival. The average time taken for issuing a ticket is 1 $min$. Assuming that customer arrivals form a Poisson process and service times are exponentially distributed, the average waiting time in queue in $min$ is
\begin{enumerate}
    \item 3
    \item 4
    \item 5
    \item 6 \\
\end{enumerate}
			 \end{enumerate}
			 \end{document}
 E
