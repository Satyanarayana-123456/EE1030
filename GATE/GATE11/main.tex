\let\negmedspace\undefined
\let\negthickspace\undefined
\documentclass[journal]{IEEEtran}
\usepackage[a5paper, margin=10mm, onecolumn]{geometry}
%\usepackage{lmodern} % Ensure lmodern is loaded for pdflatex
\usepackage{tfrupee} % Include tfrupee package

\setlength{\headheight}{1cm} % Set the height of the header box
\setlength{\headsep}{0mm}     % Set the distance between the header box and the top of the text

\usepackage{gvv-book}
\usepackage{gvv}
\usepackage{cite}
\usepackage{amsmath,amssymb,amsfonts,amsthm}
\usepackage{algorithmic}
\usepackage{graphicx}
\usepackage{textcomp}
\usepackage{xcolor}
\usepackage{txfonts}
\usepackage{listings}
\usepackage{enumitem}
\usepackage{mathtools}
\usepackage{gensymb}
\usepackage{comment}
\usepackage[breaklinks=true]{hyperref}
\usepackage{tkz-euclide} 
\usepackage{listings}
% \usepackage{gvv}                                        
\def\inputGnumericTable{}                                 
\usepackage[latin1]{inputenc}                                
\usepackage{color}                                            
\usepackage{array}                                            
\usepackage{longtable}                                       
\usepackage{calc}                                             
\usepackage{multirow}                                         
\usepackage{hhline}                                           
\usepackage{ifthen}                                           
\usepackage{lscape}
\begin{document}

\bibliographystyle{IEEEtran}
\vspace{3cm}




\title{
%	\logo{
GATE - 2014- AE

\large{EE1030 : Matrix Theory}

Indian Institute of Technology Hyderabad
%	}
}
\author{Satyanarayana Gajjarapu

AI24BTECH11009
}	





\maketitle




\bigskip

\renewcommand{\thefigure}{\theenumi}
\renewcommand{\thetable}{\theenumi}


\section{27 - 39}


\begin{enumerate}
\item For a given fuel flow rate and thermal efficiency, the take-off thrust for a gas turbine engine burning aviation turbine fuel (considering fuel-air ratio $f<<1$) is
    \begin{enumerate}
      \item Directly proportional to exhaust velocity 
      \item Inversely proportional to exhaust velocity
      \item Independent of exhaust velocity
      \item Directly proportional to the square of the exhaust velocity \\
    \end{enumerate}
\item For a fifty percent reaction axial compressor stage, following  statements are given: \\
I. Velocity triangles at the entry and exit of the rotor are symmetrical \\
II. The whirl or swirl component of absolute velocity at the entry of rotor and entry of stator are same. \\
Which of the following options are correct?
\begin{enumerate}
    \item Both I and II are correct statements 
    \item I is correct but II is incorrect
    \item I is incorrect but II is correct
    \item Both I and II are incorrect \\
\end{enumerate}
\item A small rocket having a specific impulse of $200s$ produces a total thrust of $98kN$, out of which $10kN$ is the pressure thrust. Considering the acceleration due to gravity to be $9.8m/s^2$, the propellant mass flow rate in $kg/s$ is
\begin{enumerate}
    \item 55.1 
    \item 44.9
    \item 50
    \item 60.2 \\
\end{enumerate}
\item The thrust produced by a turbojet engine
 \begin{enumerate}
     \item Increases with increasing compressor pressure ratio
     \item Decreases with increasing compressor pressure ratio
     \item Remains constant with increasing compressor pressure ratio
     \item First increases and then decreases with increasing compressor pressure ratio \\
 \end{enumerate}
\item The moment coefficient measured about the centre of gravity and about aerodynamic centre of a given wing-body combination are 0.0065 and -0.0235 respectively. The aerodynamic centre lies 0.06 chord lengths ahead of the centre of gravity. The lift coefficient for this wing-body is $\_\_\_\_$. \\
\item The vertical ground load factor on a stationary aircraft parked in its hangar is:
\begin{enumerate}
    \item 0
    \item -1
    \item Not defined
    \item 1 \\
\end{enumerate}
\item Under what condition should a glider be operated to ensure minimum sink rate?
\begin{enumerate}
    \item Maximum $\frac{C_L}{C_D}$
    \item Minimum $\frac{C_L}{C_D}$
    \item Maximum $\frac{C_D}{C_L^\frac{3}{2}}$
    \item Minimum $\frac{C_D}{C_L^\frac{3}{2}}$ \\
 \end{enumerate}
\item In most airplanes, the Dutch roll mode can be excited by applying
 \begin{enumerate}
    \item a step input to the elevators
    \item a step input to the rudder
    \item a sinusoidal input to the aileron
    \item an impulse input to the elevators \\
 \end{enumerate}
\item Considering \textbf{R} as the radius of the moon, the ratio of the velocities of two spacecraft orbiting moon in circular orbit at altitudes \textbf{R} and \textbf{2R} above the surface of the moon is $\_\_\_\_$. \\
\item If $\sbrak{A} = \sbrak{\begin{matrix}
    3 & -3 \\ -3 & 4
\end{matrix}}$. Then $\det\brak{-\sbrak{A}^2 + 7\sbrak{A} - 3\sbrak{I}}$ is
\begin{enumerate}
    \item 0
    \item -324
    \item 324 
    \item 6 \\
\end{enumerate}
\item For the periodic function given by
\begin{align*}
    f\brak{x} = \begin{cases}
        -2, & -\pi < x < 0 \\
        2, & 0 < x < \pi
    \end{cases}
\end{align*}
with $f\brak{x+2\pi} = f\brak{x}$, using Fourier series, the sum
\begin{align*}
    s = 1 - \frac{1}{3} + \frac{1}{5} - \frac{1}{7} + \cdots
\end{align*}
converges to
\begin{enumerate}
    \item 1
    \item $\frac{\pi}{3}$
    \item $\frac{\pi}{4}$
    \item $\frac{\pi}{5}$ \\
\end{enumerate}
\item Let $\Gamma$ be the boundary of the closed circular region $A$ given by $x^2 + y^2 \leq 1$. Then 
\begin{align*}
    I = \int\limits_{\Gamma}\brak{3x^3 - 9xy^2}ds
\end{align*}
(where $ds$ means integration along the bounding curve) is 
  \begin{enumerate}
   \item $\pi$
   \item -$\pi$
   \item 1
   \item 0 \\
\end{enumerate}
\item Solution to the boundary-value problem
\begin{align*}
    -9\frac{d^2u}{dx^2} + u = 5x,\ 0 < x <3
\end{align*}
with $u\brak{0} = 0$, $\frac{du}{dx}\bigg{|}_{x=3} = 0$ is
\begin{enumerate}
    \item $u\brak{x} = \frac{15e}{1+e^2}\brak{e^{-\frac{x}{3}} - e^{\frac{x}{3}}} + 5x$
    \item $u\brak{x} = \frac{15e}{1+e^2}\brak{e^{-\frac{x}{3}} + e^{\frac{x}{3}}} + 5x$
    \item $u\brak{x} = -\frac{15\sin\brak{\frac{x}{3}}}{\cos\brak{1}} + 5x$
    \item $u\brak{x} = -\frac{15\sin\brak{\frac{x}{3}}}{\cos\brak{1}} - \frac{5}{54}x^3$ \\
\end{enumerate}
			 \end{enumerate}
			 \end{document}
 
