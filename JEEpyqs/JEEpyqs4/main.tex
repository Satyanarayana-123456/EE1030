\let\negmedspace\undefined
\let\negthickspace\undefined
\documentclass[journal]{IEEEtran}
\usepackage[a5paper, margin=10mm, onecolumn]{geometry}
%\usepackage{lmodern} % Ensure lmodern is loaded for pdflatex
\usepackage{tfrupee} % Include tfrupee package

\setlength{\headheight}{1cm} % Set the height of the header box
\setlength{\headsep}{0mm}     % Set the distance between the header box and the top of the text

\usepackage{gvv-book}
\usepackage{gvv}
\usepackage{cite}
\usepackage{amsmath,amssymb,amsfonts,amsthm}
\usepackage{algorithmic}
\usepackage{graphicx}
\usepackage{textcomp}
\usepackage{xcolor}
\usepackage{txfonts}
\usepackage{listings}
\usepackage{enumitem}
\usepackage{mathtools}
\usepackage{gensymb}
\usepackage{comment}
\usepackage[breaklinks=true]{hyperref}
\usepackage{tkz-euclide} 
\usepackage{listings}
% \usepackage{gvv}                                        
\def\inputGnumericTable{}                                 
\usepackage[latin1]{inputenc}                                
\usepackage{color}                                            
\usepackage{array}                                            
\usepackage{longtable}                                       
\usepackage{calc}                                             
\usepackage{multirow}                                         
\usepackage{hhline}                                           
\usepackage{ifthen}                                           
\usepackage{lscape}
\begin{document}

\bibliographystyle{IEEEtran}
\vspace{3cm}




\title{
%	\logo{
28-07-2022 Shift-1

\large{EE1030 : Matrix Theory}

Indian Institute of Technology Hyderabad
%	}
}
\author{Satyanarayana Gajjarapu

AI24BTECH11009
}	





\maketitle




\bigskip

\renewcommand{\thefigure}{\theenumi}
\renewcommand{\thetable}{\theenumi}


\section{\large Shift-1(16-30)}


\begin{enumerate}
\item The foot of the perpendicular from a point on the circle $x^2 + y^2 = 1$, $z = 0$ to the plane $2x + 3y + z = 6$ lies on which one of the following curves?
\begin{enumerate}
    \item $\brak{6x + 5y - 12}^2 + 4\brak{3x + 7y - 8}^2 = 1$, $z = 6 - 2x - 3y$
    \item $\brak{5x + 6y - 12}^2 + 4\brak{3x + 5y - 9}^2 = 1$, $z = 6 - 2x - 3y$
    \item $\brak{6x + 5y - 14}^2 + 9\brak{3x + 5y - 7}^2 = 1$, $z = 6 - 2x - 3y$
    \item $\brak{5x + 6y - 14}^2 + 9\brak{3x + 7y - 8}^2 = 1$, $z = 6 - 2x - 3y$\\
\end{enumerate}
\item If the minimum value of 
\begin{align*}
    f\brak{x} = \frac{5x^2}{2} + \frac{\alpha}{x^5}, x>0
\end{align*}
is 14, then the value of $\alpha$ is equal to
  \begin{enumerate}
      \item 32
      \item 64
      \item 128
      \item 256\\
  \end{enumerate}
\item Let $\alpha, \beta, \gamma$ be three positive real numbers. Let $f\brak{x} = \alpha x^5 + \beta x^3 + \gamma x, x \in \mathbb{R}$ and $g:\mathbb{R}\rightarrow\mathbb{R}$ be such that $g\brak{f\brak{x}} = x$ for all $x\in\mathbb{R}$. If $a_1, a_2, a_3, \cdots, a_n$ be in arithmetic progression with mean zero, then the value of 
\begin{align*}
    f\brak{g\brak{\frac{1}{n}\sum_{i=1}^{n}f\brak{a_i}}}
\end{align*}
is equal to
     \begin{enumerate}
         \item 0
         \item 3
         \item 9
         \item 27\\
     \end{enumerate}
\item Consider the sequence $a_1, a_2, a_3, \cdots$ such that $a_1$ = 1, $a_2$ = 2 and $a_{n+2} = \frac{2}{a_{n+1} + a_n}$ for $n = 1, 2, 3, \cdots$. If 
\begin{align*}
    \brak{\frac{a_1 + \frac{1}{a_2}}{a_3}}\brak{\frac{a_2 + \frac{1}{a_3}}{a_4}}\brak{\frac{a_3 + \frac{1}{a_4}}{a_5}}\cdots\brak{\frac{a_{30} + \frac{1}{a_{31}}}{a_{32}}} = 2^{\alpha}\brak{^{61}C_{31}},
\end{align*}
then $\alpha$ is equal to 
 \begin{enumerate}
     \item -30
     \item -31
     \item -60
     \item -61\\
 \end{enumerate}
\item The minimum value of the twice differentiable function
\begin{align*}
    f\brak{x} = \int_{0}^{x}e^{x-t}f'\brak{t} - \brak{x^2 - x + 1}e^{x}, x\in\mathbb{R}
\end{align*}
is
\begin{enumerate}
    \item $-\frac{2}{\sqrt{e}}$
    \item $-2\sqrt{e}$
    \item $-\sqrt{e}$
    \item $\frac{2}{\sqrt{e}}$\\
\end{enumerate}
\item Let $S$ be the set of all passwords which are six to eight chaacters long, where each character is either an alphabet from $\{A, B, C, D, E\}$ or a number from $\{1, 2, 3, 4, 5\}$ with the repetition of characters allowed. If the number of passwords in $S$ whose at least one character is a number from $\{1, 2, 3, 4, 5\}$ is $\alpha \times 5^6$, then $\alpha$ is equal to \_\_\_\_\_\_ \\
\item Let $P\brak{-2, -1, 1}$ and $Q\brak{\frac{56}{17}, \frac{43}{17}, \frac{111}{17}}$ be the vertices of the rhombus $PRQS$. If the direction ratios of the diagonal $RS$ are $\alpha$, -1, $\beta$, where both $\alpha$ and $\beta$ are integers of minimum absolute values, then $\alpha^2 + \beta^2$ is equal to \_\_\_\_\_\_ \\
\item Let $f:\sbrak{0, 1}\rightarrow\textbf{R}$ be a twice differentiable function in \brak{0, 1} such that $f\brak{0} = 3$ and $f\brak{1} = 5$. If the line $y = 2x + 3$ intersects the graph of $f$ at only two distinct points in \brak{0, 1} then the least number of points $x\in\brak{0, 1}$ at which $f''\brak{x} = 0$, is \_\_\_\_\_\_ \\
\item If 
\begin{align*}
\int_{0}^{\sqrt{3}}\frac{15x^3}{\sqrt{1+x^2+\sqrt{\brak{1+x^2}^3}}}dx= \alpha\sqrt{2} + \beta\sqrt{3}
\end{align*}
where $\alpha$, $\beta$ are integers, then $\alpha+\beta$ is equal to \\
\item Let $A=\sbrak{\begin{matrix}
    1 & -1 \\ 2 & \alpha
\end{matrix}}$ and $B=\sbrak{\begin{matrix}
    \beta & 1 \\ 1 & 0
\end{matrix}}$, $\alpha, \beta \in R$. Let $\alpha_1$ be the value of $\alpha$ which satisfies 
\begin{align*}
    \brak{A+B}^2 = A^2 + \sbrak{\begin{matrix}
        2 & 2 \\ 2 & 2
    \end{matrix}}
\end{align*}
and $\alpha_2$ be the value of $\alpha$ which satisfies
\begin{align*}
    \brak{A+B}^2 = B^2.
\end{align*}
Then $\abs{\alpha_1 - \alpha_2}$ is equal to \_\_\_\_\_\_ \\
\item For $p, q \in R$, consider the real valued function $f\brak{x} = \brak{x - p}^2 - q, x \in R$ and $q > 0$. Let $a_1$, $a_2$, $a_3$ and $a_4$ be in an arithmetic progression with mean $p$ and positive common difference. If $\abs{f\brak{a_i}}$ = 500 for all $i = 1, 2, 3, 4$, then the absolute difference between the roots of $f\brak{x} = 0$ is \\
\item For the hyperbola $H: x^2 - y^2 = 1$ and the ellipse $E: \frac{x^2}{a^2} + \frac{y^2}{b^2} = 1, a > b > 0$, let the
\begin{enumerate}
    \item eccentricity of $E$ be reciprocal of the eccentricity of $H$, and
    \item the line $y = \sqrt{\frac{5}{2}} x + K$ be a common tangent of $E$ and $H$.
\end{enumerate}
Then 4\brak{a^2 + b^2} is equal to \_\_\_\_\_\_ \\
\item Let $x_1, x_2, x_3, \cdots, x_{20}$ be in geometric progression with $x_1 = 3$ and the common ratio $\frac{1}{2}$. A new data is constructed replacing each $x_i$ by $\brak{x_i - i}^2$. If $\overline{x}$ is the mean of new data, then the greatest integer less than or equal to $\overline{x}$ is  \_\_\_\_\_\_ \\
\item 
\begin{align*}
    \lim\limits_{x \rightarrow 0}\brak{\frac{\brak{x+2\cos\brak{x}}^3 + 2\brak{x+2\cos\brak{x}}^2 + 3\sin\brak{x+2\cos\brak{x}}}{\brak{x+2}^3 + 2\brak{x+2}^2 + 3\sin\brak{x+2}}}^\frac{100}{x}
\end{align*}
is equal to \_\_\_\_\_\_ \\
\item The sum of all real value of $x$ for which
\begin{align*}
    \frac{3x^2-9x+17}{x^2+3x+10} = \frac{5x^2-7x+19}{3x^2+5x+12}
\end{align*}
is equal to \_\_\_\_\_\_ \\
			 \end{enumerate}
			 \end{document}

